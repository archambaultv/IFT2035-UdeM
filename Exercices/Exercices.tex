%%%% Recueil d'exercice pour IFT 2035
%%%% Vincent Archambault-Bouffard
%%%% vincent.archambault-bouffard@umontreal.ca

%%%% This document is dedicated to the public domain via CC0

%%%% You must compile this document with XeTeX

\documentclass{scrreprt}

%%%% Date, title, author %%%%
\date{\small Version du \today} 
\title{Université de Montréal\\IFT 2035 - Exercices}
\author{Vincent Archambault-Bouffard}

% See : https://tex.stackexchange.com/questions/8351/what-do-makeatletter-and-makeatother-do
% \thetitle now contains the title, same for author and date
\makeatletter
\let\thetitle\@title
\let\theauthor\@author
\let\thedate\@date
\makeatother

%%%% Math  %%%%
\usepackage{amsmath} % should go before unicode-math
\usepackage{unicode-math} % loads fontspec package

%%%% Fonts  %%%%
\setmainfont{Latin Modern Roman}
\setsansfont{Latin Modern Sans}
\setmonofont{Latin Modern Mono}
\setmathfont{Latin Modern Math}

%%%% Languages %%%%
\usepackage{polyglossia}
\setmainlanguage{french}

\makeatletter
\newcommand{\nofrenchpunctuation}[1]{\nofrench@punctuation {#1}\french@punctuation}
\makeatother

%%%% Colors %%%%
\usepackage{xcolor}

%%%% Tables %%%%
\usepackage{booktabs}

%%%% Hyperlinks %%%%
\usepackage{hyperref}
% ex : \href{https://www.wikibooks.org}{Wikibooks home}
\hypersetup{
    pdftitle={UdeM - IFT 2035 - Exercices},    % title
    pdfauthor={\theauthor},     % author
    colorlinks=true,       % false: boxed links; true: colored links
    linkcolor=black,          % color of internal links 
    citecolor=black,        % color of links to bibliography
    filecolor=magenta,      % color of file links
    urlcolor=black           % color of external links
}

%%%% Source Code %%%%
% Requires -shell-escape option
% Requires python and the Pygments package
\usepackage{minted}
\usemintedstyle{friendly} %colorful emacs bw friendly
\newcommand{\codeinline}[2][]{\mintinline[#1]{text}{#2}}
\newcommand{\codeinput}[2][]{\inputminted[#1]{text}{#2}}
\newcommand{\haskellinline}[2][]{\mintinline[#1]{haskell}{#2}}
\newcommand{\haskellinput}[2][]{\inputminted[#1]{haskell}{#2}}
\newcommand{\cinline}[2][]{\mintinline[#1]{c}{#2}}
\newcommand{\cinput}[2][]{\inputminted[#1]{c}{#2}}
\newcommand{\assemblyinline}[2][]{\mintinline[#1]{gas}{#2}}
\newcommand{\assemblyinput}[2][]{\inputminted[#1]{gas}{#2}}

%%%% Modify enumeration %%%%
\usepackage{enumitem}

%%%% Comment to be turned off or on %%%%
\usepackage{comment}
%\excludecomment{solution}
%\includecomment{solution}


%%%% Proof Tree %%%%
\usepackage{bussproofs}

%%%% Headers %%%%
\usepackage{fancyhdr}
%\pagestyle{fancy}
% \fancyhf{}
% \rhead{Share\LaTeX}
% \lhead{Guides and tutorials}
% \rfoot{Page \thepage}
% \renewcommand{\headrulewidth}{2pt}
% \renewcommand{\footrulewidth}{1pt}

\usepackage[answerdelayed]{exercise}
\counterwithin{Exercise}{chapter}
\counterwithin{Answer}{chapter}
\renewcounter{Exercise}[chapter]
\renewcommand{\ExerciseName}{Exercice}
\renewcommand{\AnswerName}{Solution de l'exercice}
\renewcommand{\ExePartName}{Partie}
\renewcommand{\ExerciseHeader}{\centerline{\textbf{\large\ExerciseName\ \ExerciseHeaderNB\medskip}}}
%\renewcommand{\ExerciseHeader}{\centerline{\textbf{\large\ExerciseName\ \ExerciseHeaderNB\ExerciseHeaderTitle\ExerciseHeaderOrigin\medskip}}}
\renewcommand{\ExePartHeader}{\vbox{\addvspace{\bigskipamount}\hbox{\textbf{\ExePartName\;\ExePartHeaderNB\ExePartHeaderTitle\medskip}}}}
\begin{document}

\maketitle

Ce document est dédié au domaine public via \href{https://creativecommons.org/publicdomain/zero/1.0/}{CC0 1.0}.

\bigskip
Pour obtenir le code source de ce document :
\begin{itemize}
\item \href{https://github.com/archambaultv/IFT2035-UdeM}{\texttt{\nofrenchpunctuation{https://github.com/archambaultv/IFT2035-UdeM}}}
\item \texttt{vincent.archambault-bouffard@umontreal.ca}
\end{itemize}

\pagebreak

\chapter*{Remerciement}
Je tiens à remercier les personnes suivantes (par ordre alphabétique) :
\begin{description}
  \item[Frédéric Hamel] Pour avoir fourni plusieurs solutions.
  \item[Stefan Monnier] Pour avoir fourni plusieurs exercices.
  \end{description}


\chapter{Introduction}
Petit recueil d'exercices pour IFT 2035. Vous trouverez les solutions à la fin, lorsqu'elles sont disponibles. 

\chapter{Grammaire}

\begin{Exercise}[title={Conversion préfixe, infixe, postfixe}]
  \label{ex:infixe-prefixe-postfixe}
Pour chaque expression infixe ci-dessous, réécriver l'expression en
notation préfixe et postfixe. Dessiner également l'arbre de syntaxe
abstraite (ASA).

\begin{enumerate}
\item $a + b + c$
\item $a + (b + c)$
\item $a \cdot b + c \cdot d$
\item $a + b < a \cdot (c + d)$
\item $\sqrt{b \cdot b - 4 \cdot a \cdot c}$
\end{enumerate}
\end{Exercise}

\begin{Answer}[ref={ex:infixe-prefixe-postfixe}]
La réponse contient d'abord l'expression en préfixe, ensuite l'expression en postfixe.
\begin{enumerate}
\item $+ + a b c$ et $a b + c +$
\item $+ a + b c$ et $a b c + +$
\item $+ \cdot a b \cdot c d$ et $a b \cdot c d \cdot +$ 
\item $< + a b \cdot a + c d$ et $a b + a c d + \cdot <$
\item $\sqrt{} - \cdot b b \cdot \cdot 4 a c$ et $b b \cdot 4 a \cdot c \cdot - \sqrt{}$
\end{enumerate}
\end{Answer}


\begin{Exercise}[title={Grammaire pour l'addition et la multiplication}]
  \label{ex:grammaire-add-mult}
Nous allons voir dans cette section plusieurs grammaires BNF
possibles pour l'addition et la multiplication. Vous verrez qu'il est
facile de générer une grammaire BNF ambiguë ou ne respectant pas la
priorité et l'associativité des opérations.

\ExePart[title={Grammaire 1}]
\label{expart:grammaire1}
Soit la grammaire ci-dessous:
\begin{align*}
  \text{<expr>}  &\text{::=  <expr> + <expr> | <chiffre>} \\
  \text{<chiffre>}  &\text{::=  0 | 1 | 2 | 3 | 4 | 5 | 6 | 7 | 8 | 9 }
\end{align*}
\begin{enumerate}
\item Montrer que dans cette grammaire, pour une même expression, il
  est possible de choisir une dérivation où l'opérateur est
  associatif à gauche et une dérivation ou l'opérateur est
  associatif à droite. Cela démontre que la grammaire est ambiguë.
\end{enumerate}
\vspace{-\bigskipamount}

\ExePart[title={Grammaire 2}]
Voici comment on pourrait transformer la grammaire 1 pour lever
l'ambiguïté.
\begin{align*}
  \text{<expr>}  &\text{::=  <expr> + <terme> | <chiffre>} \\
  \text{<terme>} &\text{::= '(' <expr> ')' | <chiffre>} \\
  \text{<chiffre>}  &\text{::=  0 | 1 | 2 | 3 | 4 | 5 | 6 | 7 | 8 | 9 }
\end{align*}
\begin{enumerate}
\item Montrer que cette grammaire est associative à gauche.
\item Donner l'arbre de dérivation de l'expression $1 + (1 + 1) + 1$.
\end{enumerate}
Noter que cette grammaire est récursive à gauche et associative à
gauche.

\ExePart[title={Grammaire 3}]
Donner une grammaire BNF similaire à la grammaire 1 mais cette
fois-ci associative à droite.

\ExePart[title={Grammaire 4}]
La grammaire 2 est associative à gauche et non ambiguë. Voici une
première tentative d'ajouter l'opérateur de multiplication.
\begin{align*}
  \text{<expr>}  &\text{::=  <expr> + <terme> |  <expr> * <terme> | <chiffre>} \\
  \text{<terme>} &\text{::= '(' <expr> ')' | <chiffre>} \\
  \text{<chiffre>}  &\text{::=  0 | 1 | 2 | 3 | 4 | 5 | 6 | 7 | 8 | 9 }
\end{align*}

\begin{enumerate}
\item Montrer que cette grammaire est non ambiguë.
\item Cette grammaire ne respecte pas la priorité des
  opérations. Donner l'arbre de dérivation des expressions
  $1 + 2 * 3$ et $1 * 2 + 3$.
\end{enumerate}
\vspace{-\bigskipamount}

\ExePart[title={Grammaire 5}]
Soit la grammaire ci-dessous qui corrige le défaut de la grammaire 4. 
\begin{align*}
  \text{<expr>}  &\text{::=  <expr> + <terme> |  <terme>} \\
  \text{<terme>} &\text{::=  <terme> * <facteur> | <facteur>} \\
  \text{<facteur>} &\text{::= '(' <expr> ')' | <chiffre>} \\
  \text{<chiffre>}  &\text{::=  0 | 1 | 2 | 3 | 4 | 5 | 6 | 7 | 8 | 9 }
\end{align*}

\begin{enumerate}
\item Cette grammaire respecte la priorité des opérations. Donner
  l'arbre de dérivation des expressions $1 + 2 * 3$ et $1 * 2 + 3$.
\end{enumerate}
\vspace{-\bigskipamount}

\ExePart[title={Grammaire 6}]
Voici comment ajouter la soustraction et la division. 
\begin{align*}
  \text{<expr>}  &\text{::=  <expr> + <terme> | <expr> - <terme> |  <terme>} \\
  \text{<terme>} &\text{::=  <terme> * <facteur> |  <terme> / <facteur> | <facteur>} \\
  \text{<facteur>} &\text{::= '(' <expr> ')' | <chiffre>} \\
  \text{<chiffre>}  &\text{::=  0 | 1 | 2 | 3 | 4 | 5 | 6 | 7 | 8 | 9 }
\end{align*}

\begin{enumerate}
\item Pourquoi peut-on ajouter la soustraction (division) dans la même
  catégorie que l'addition (multiplication) plutôt que de créer une
  nouvelle catégorie ?
\end{enumerate}
\vspace{-\bigskipamount}

\ExePart[title={Grammaire 7 - Pour les curieux}]
La grammaire 5 est non ambiguë et respecte la priorité et
l'associativité des opérateurs. Toutefois, elle est récursive à gauche
pour la catégorie <expr> et <terme>. Une grammaire récursive à gauche peut
engendrer des boucles infinies. 

En effet, dans un parseur avec analyse descendante (top down), la
portion du programme devant lire la catégorie <expr> doit d'abord faire
appel à la portion du programme qui doit lire la catégorie <expr> qui doit
d'abord faire appel à la portion du programme qui doit lire la catégorie
<expr> qui doit d'abord faire appel ...

Il faut donc transformer cette grammaire en une grammaire équivalente
mais non récursive à gauche.
\begin{align*}
  \text{<expr>}  &\text{::=  <terme> <expr'>} \\
  \text{<expr'>}  &\text{::=  + <terme> <expr'> | } \epsilon \\
  \text{<terme>} &\text{::=  <facteur> <terme'>} \\
  \text{<terme'>} &\text{::=  * <facteur> <terme'> | } \epsilon \\
  \text{<facteur>} &\text{::= '(' <expr> ')' | <chiffre>} \\
  \text{<chiffre>}  &\text{::=  0 | 1 | 2 | 3 | 4 | 5 | 6 | 7 | 8 | 9 }
\end{align*}

Il est possible de montrer que cette grammaire est équivalente à la
grammaire du numéro 5.
\end{Exercise}

\begin{Exercise}
  \label{ex:if_then_else}
Voici une grammaire pour if then else. Les règles <E> et <X>
représente des expressions et autres règles de la grammaire dont il
n'est pas important de spécifier.
\begin{minted}{text}
  <S> ::=  <X>
       | 'if' <E> 'then' <S>
       | 'if' <E> 'then' <S> 'else' <S>} 
\end{minted}

Cette grammaire est ambiguë. 
\begin{enumerate}
\item Donner un exemple d'ambiguïté.
\item Donner la grammaire non ambiguë qui associe les else avec le if
  le plus proche.
\end{enumerate}
\end{Exercise}

\begin{Answer}[ref={ex:if_then_else}]
  Voici une phrase ambigüe : \codeinline{if E1 then if E2 then E3 else E4}. Elle
  peut s'interpréter des deux façons ci-dessous.

  \begin{itemize}
  \item \codeinline{if E1 then (if E2 then E3 else E4)}
  \item \codeinline{if E1 then (if E2 then E3) else E4}.
  \end{itemize}
    
  Pour lever l'ambiguïté, il faut par exemple arbitrairement décider que les
  \codeinline{else} se réfèrent aux \codeinline{if} les plus proches.

\begin{minted}{text}
<S> ::= <X>
    | 'if' <E> 'then' <S>
    | 'if' <E> 'then' <SElse> 'else' <S>

<SElse> ::=
  | 'if' <E> 'then' <SElse> 'else' <SElse>
\end{minted}
\end{Answer}

\chapter{Variables}

\input{exercices/variable_portée1.tex}

\input{exercices/implantation_portée.tex}

\chapter{Fonctions}

\begin{Exercise}
  \label{ex:passage_portee}
Écrire un programme qui donne un résultat différent pour chacune des
combinaisons possibles entre les 4 méthodes de passage de paramètres et
les 2 types de portées (8 combinaisons en tout). Pour rappel, les 4
types de passage de paramètres sont :
\begin{itemize}
\item passage de paramètres par valeur 
\item passage de paramètres par référence
\item passage de paramètres par valeur-résultat
\item passage de paramètres par nom
\end{itemize}
\end{Exercise}

\begin{Answer}[ref={ex:passage_portee}]
  Il est plus facile de scinder le problème en deux. Une fonction qui teste la
  portée et l'autre le passage des paramètres. Voici un programme C qui permet
  de faire la différence.
  
  \cinput{exercices/passage_paramètres_portée.c}

\end{Answer}

\begin{Exercise}
\label{ex:raisonner}
  Soit le morceau de code suivant dans un langage hypothétique qui
utilise une syntaxe de style C, et où \codeinline{f} est une fonction
quelconque \emph{que l'on ne connaît pas}:
\begin{minted}{C}
{
  int table[2] = {0, 1};
  int size = 2;
  int tmp = 0;

  f (table, size);
  ...
}
\end{minted}
On aimerait savoir si certaines conditions sont nécessairement
toujours vraies aprés l'appel à \codeinline{f}.  On s'intéresse plus
particulièrement aux conditions suivantes:
\begin{itemize}
\item \codeinline{table[0] == 0}
\item \codeinline{size == 2}
\item \codeinline{tmp == 0}
\end{itemize}
Indiquer lesquelles de ces trois conditions sont nécessairement vraies
dans chacun des cas suivants:
\begin{enumerate}
\item Le langage est exactement comme C: portée statique, passage
  d'arguments par valeur, affectation autorisée.
\item Le langage est comme C sauf que l'affectation (autre que
  l'initialization) est interdite.
\item Le langage est comme C sauf que les arguments sont passés par
  référence.
\item Le langage est comme C mais avec portée dynamique.
\end{enumerate}
\end{Exercise}

% \begin{Answer}[ref={ex:raisonner}]

% \end{Answer}

\chapter{Récursion}
\input{exercices/récursion1.tex}

\chapter{Fermeture}
\begin{Exercise}
\label{ex:datatype_fermeture}
Cet exercice a pour objectif de vous faire pratiquer les fermetures et de
vous convaincre qu'elles peuvent servir de structure de données. Nous
allons utiliser les encodages de Church pour écrire les structures de
données usuelles sous forme de fonctions. Chaque fonction peut être
écrite sur une ligne, mais ce sont des lignes qui ne sont pas toujours
facile à trouver.

\subsection*{Détail technique}
Nous allons redéfinir plusieurs identificateurs de la libraire
standard. Je vous conseille de mettre la ligne suivant au début de
votre fichier Haskell pour éviter que le compilateur ou ghci vous
indique que vous faites ombrage aux définitions standards.
\begin{minted}{haskell}
import Prelude hiding (Bool, succ, fst, snd, head, tail)
\end{minted}

\ExePart[title={Booléen}]
\label{sec:boolean}
Vous devez implanter les booléens sous forme de fonctions. Pour vous
donner un indice, le type des deux fonctions qui représentent le
boolen \codeinline{true} et \codeinline{false} est le suivant:
\begin{minted}{haskell}
type Bool a = a -> a -> a
\end{minted}

Vous devez :
\begin{enumerate}
\item Écrire la fonction \codeinline{true} et \codeinline{false} de
  type \codeinline{Bool a}. En fait, il n'y a que deux fonctions
  possibles avec ce type.
\item Écrire la fonction \codeinline{if2} qui a pour type
  \codeinline{Bool a -> a -> a -> a}. C'est à dire quelle prend un
  booléen et deux valeur de type \codeinline{a} et retourne la première
  valeur si le booléen est \codeinline{true} et la deuxième sinon.
\end{enumerate}

Par exemple, dans le code suivant \codeinline{testIfTrue} vaut 1
\begin{minted}{haskell}
testIfTrue = if2 true 1 2
\end{minted}
et dans le code suivant \codeinline{testIfFalse} vaut 2
\begin{minted}{haskell}
testIfFalse = if2 false 1 2
\end{minted}

\ExePart[title={Nombres naturels}]
\label{sec:nombre_naturel}
Vous devez implanter les nombres naturel $0 \ 1 \ 2 \ 3 \ldots$ à l'aide de
fonctions. Le type d'un nombre naturel est :
\begin{minted}{haskell}
type Number t = (t -> t) -> t -> t
\end{minted}

La logique de l'encodage est la suivante : un nombre est représenté
par une fonction qui attend une valeur de base de type \codeinline{t}
et une fonction de type \codeinline{t -> t}. Remarquez que la fonction
retourne une valeur de type \codeinline{t}. L'idée est que 0
correspond à la valeur de base, 1 correspond à \emph{une} application
de la fonction, 2 correspond à \emph{deux} applications de la fonction,
3 correspond à \emph{trois} applications de la fonction, etc.

Vous devez :
\begin{enumerate}
\item Écrire la fonction \codeinline{zero} de type \codeinline{Number
    t} qui correspond au nombre 0.
\item Écrire la fonction \codeinline{succ} de type 
    \codeinline{Number t -> Number t} qui retourne le successeur du nombre
    reçu en argument. Pour vous aider, voici une définition partielle de 
    \codeinline{succ}
\begin{minted}{haskell}
succ :: Number t -> Number t
succ n = \f z ->
\end{minted}
  \item Écrire la fonction \codeinline{plus} de type
    \codeinline{Number t -> Number t -> Number t} qui prend deux
    entiers et retourne un entier qui représente l'addition des deux.
\end{enumerate}

Par exemple, dans le code suivant \codeinline{myOne} vaut 1
\begin{minted}{haskell}
one :: Number t
one = succ zero

myOne = one (+ 1) 0
\end{minted}
et dans le code suivant \codeinline{myTwo} vaut 10
\begin{minted}{haskell}
two :: Number t
two = plus (succ zero) (succ zero)

myTwo = two (+ 5) 0
\end{minted}

\ExePart[title={Paires}]
\label{sec:paires}
Vous devez implanter le concept d'une paire à l'aide de fonctions. Le
type d'une paire est :
\begin{minted}{haskell}
type Pair a b t = (a ->  b ->  t) ->  t
\end{minted}

Vous devez :
\begin{enumerate}
\item Écrire la fonction \codeinline{mkPair} de type
  \codeinline{a -> b -> Pair a b t} qui construit une
  paire. Pour vous aider, voici une définition partielle de
  \codeinline{mkPair}
\begin{minted}{haskell}
mkPair :: a -> b -> Pair a b t
mkPair a b = \f -> 
\end{minted}

\item Écrire la fonction \codeinline{fst} de type \codeinline{Pair a b
    a -> a} qui retourne le premier élément de la paire.

\item Écrire la fonction \codeinline{snd} de type \codeinline{Pair a b
    b -> b} qui retourne le deuxième élément de la paire.
\end{enumerate}

Par exemple, dans le code suivant \codeinline{x} vaut 0
\begin{minted}{haskell}
x = fst (mkPair 0 "c")
\end{minted}
et dans le code suivant \codeinline{x} vaut \codeinline{"c"}
\begin{minted}{haskell}
x = snd (mkPair 0 "c")
\end{minted}

\ExePart[title={Liste}]
\label{sec:paires}
Finalement, vous devez implanter le concept d'une liste à l'aide de fonctions. Le
type d'une liste est :
\begin{minted}{haskell}
type List a t = (a->t->t)->t->t
\end{minted}
Ce type est très similaire à celui des entiers. Vous pouvez réutiliser
la même logique.

Vous devez :
\begin{enumerate}
\item Écrire la fonction \codeinline{cons} de type \codeinline{a ->
    List a t -> List a t}. Pour vous aider, voici une définition
  partielle de \codeinline{cons}
\begin{minted}{haskell}
cons :: a -> List a t -> List a t
cons x l = \f z ->
\end{minted}

\item Écrire la fonction \codeinline{nil} de type \codeinline{List a
    t} qui représente la liste vide.

\item Écrire la fonction \codeinline{isNil} de type \codeinline{List a
    (Bool t2) -> Bool t2} qui retourne vrai ou faux si la liste
  est vide ou non. Il s'agit bien du type \codeinline{Bool a} vu à la
  section \ref{sec:boolean}

\item Écrire la fonction \codeinline{head} de type \codeinline{List a
    a -> a -> a} qui retourne le premier élément de la liste. Le
  deuxième paramètre est une valeur par défaut qu'il faut retourner si
  la liste est vide.

\end{enumerate}

Par exemple, dans le code suivant \codeinline{x} vaut \codeinline{"vrai"}
\begin{minted}{haskell}
x =  isNil nil "vrai" "faux"
\end{minted}
et dans le code suivant \codeinline{x} vaut \codeinline{"faux"}
\begin{minted}{haskell}
x =  isNil (cons 1 nil) "vrai" "faux"
\end{minted}
et dans le code suivant \codeinline{x} vaut 1
\begin{minted}{haskell}
x =  head (cons 1 nil) 0
\end{minted}
et dans le code suivant \codeinline{x} vaut 0
\begin{minted}{haskell}
x =  head nil 0
\end{minted}

\subsection*{Pour les curieux}
Maintenant vous savez qu'en présence des fermetures il n'est pas
nécessaire (en théorie) d'avoir des structures de données. C'est
pourquoi le lambda calcul, qui est un langage minimaliste composé
uniquement de fonctions, est souvent utilisé pour modéliser les
langages fonctionnels.
\end{Exercise}

\begin{Answer}[ref={ex:datatype_fermeture}]
  \haskellinput{exercices/datatype_fermeture_solution.hs}
\end{Answer}

\begin{Exercise}[title={Tables associatives en utilisant des fermetures}]
  \label{ex:tables_fermetures}
  Définir en Haskell des fonctions pour gérer des tables associatives
sans utiliser de constructeur.  Plus précisément, définir:
\begin{minted}{haskell}
-- Le type des tables
type T a b = ... 

-- Une table vide
empty :: T a b   

-- Ajoute une valeur de type b liée à la clé de type a
-- dans une table existante
add :: (Eq a) => a -> b -> T a b -> T a b 

-- Renvoie la valeur de type b liée à la clé de type a 
-- dans la table ou sinon une valeur par défaut (3e paramètre)
lookup :: (Eq a) => a -> T a b -> b -> b 
\end{minted}

Vu que les constructeurs de données ne peuvent pas être utilisés, les
tables seront nécessairement représentées par des fermetures.
\end{Exercise}

\begin{Answer}[ref={ex:tables_fermetures}]
  \haskellinput{exercices/fermetures_solution.hs}
\end{Answer}

\chapter{Types}

\begin{Exercise}[title={Inférence de type}]
\label{inference_type}
  Inférer les types des expressions Haskell ci-dessous. Vous pouvez
tenir pour acquis que les nombres sont de type
\haskellinline{Int}.
\begin{enumerate}
\item \haskellinline{5}
\item \haskellinline{'c'}
\item \haskellinline{"Hello World"}
\item \haskellinline{[1, 2]}
\item \haskellinline{['c', 'b']}
\item \haskellinline{\x -> x + 1}
\item \haskellinline{let x = 5 in x + 1}
\item \haskellinline{\f x -> f x x}
\item \haskellinline{\f f2 y -> f (f2 y)}
\item \haskellinline{\f -> let x =5; y = "hello" in f x y}
\item \haskellinline{[\x -> x + 1, \y -> y]}
\end{enumerate}
\end{Exercise}

\begin{Answer}[ref={inference_type}]
\begin{enumerate}
\item \haskellinline{Int}
\item \haskellinline{Char}
\item \haskellinline{[Char]}
\item \haskellinline{[Int]}
\item \haskellinline{[Char]}
\item \haskellinline{Int -> Int}
\item \haskellinline{Int}
\item \haskellinline{(a -> a -> b) -> a -> b}
\item \haskellinline{(a -> b) -> (c -> a) -> c -> b}
\item \haskellinline{(Int -> [Char] -> b) -> b}
\item \haskellinline{[Int -> Int]}
\end{enumerate}
\end{Answer}

\chapter{Programmation fonctionnelle}

\input{exercices/récursion2.tex}

\begin{Exercise}[title={Calculatrice Haskell}]
Écrire une petite calculatrice Haskell qui fait des additions.
Le langage arithmétique de cette calculatrice correspond au datatype suivant:
\begin{minted}{Haskell}
data Exp = Enum Int
         | Eplus Exp Exp
\end{minted}

Il faut écrire la fonction eval qui convertit ces expressions en nombre. Cette fonction à la signature suivante :
\begin{minted}{Haskell}
eval :: Exp -> Int
\end{minted}

Voici trois exemples :
\begin{itemize}
\item 
\begin{minted}{Haskell}
eval (Enum 1) == 1
\end{minted}

\item 
\begin{minted}{Haskell}
eval (Eplus (Enum 1) (Enum 1)) == 2
\end{minted}

\item 
\begin{minted}{Haskell}
eval (Eplus (Eplus (Enum 1) (Enum 1)) (Enum 1)) == 3
\end{minted}
\end{itemize}
\end{Exercise}

\begin{Exercise}[title={Quicksort en Haskell}]
  \label{ex:quicksort_haskell}
Implanter en Haskell une variante de quicksort pour des listes
d'entiers.  En clair, trier une liste comme suit:
\begin{enumerate}
\item choisir un élément, que l'on nommera le pivot.
\item partitionner la liste en deux sous-listes d'éléments plus
  petits et respectivement plus grands que le pivot.
\item trier les deux sous-listes.
\item combiner ces sous-listes triées et le pivot en une liste triée.
\end{enumerate}
Le type sera: \haskellinline{quicksort :: [Int] -> [Int]}. Il faudra
peut-être définir une ou plusieurs fonctions auxiliaires. L'opération
de concaténation de deux listes s'écrit \haskellinline{++} en Haskell.

Finalement, généraliser la fonction de tri précédente pour pouvoir
l'appliquer à des listes quelconques (pas seulement
\haskellinline{Int}), en passant un argument supplémentaire qui
indique l'opération de comparaison à utiliser.

Donner aussi le type de cette fonction plus générale et de toutes les
fonctions auxiliaires que vous avez définies.
\end{Exercise}

\begin{Answer}[ref={ex:quicksort_haskell}]
  \haskellinput{exercices/quicksort_Haskell_solution.hs}
\end{Answer}

\begin{Exercise}[title={Table associative}]
  \label{ex:treemap}
Soit le type suivant en Haskell qui définit un arbre binaire que l'on peut
utiliser pour représenter une table associative (qui associe des \emph{clés}
de type \haskellinline{Int} à des valeurs de type \codeinline{b}):
%%
\begin{minted}{haskell}
data TreeMap b = Empty | Node Int b (TreeMap b) (TreeMap b)
\end{minted}
%%
L'exercice est de définir les opérations typiques sur une telle
structure de donnée.  Bien sûr, pour être utile l'arbre doit être
maintenu dans l'ordre: toutes les clés dans la branche de gauche d'un
\mintinline{haskell}{Node} doivent être plus petites que la clé du noeud, et
vice versa pour la branche de droite.

Il y a trois opérations:
\begin{itemize}
\item \haskellinline{tmLookup}: rechercher la valeur associée à une clé passée
  en paramètre.
\item \haskellinline{tmInsert}: ajouter une entrée (donnée sous la forme
  d'une clé et de sa valeur) dans la table.
\item \haskellinline{tmRemove}: enlever une entrée (dont la clé est passée
  en paramètre).
\end{itemize}
Ces fonctions ne doivent jamais signaler d'erreur.
\begin{enumerate}
\item Donner le type de ces trois fonctions.
\item Donner une liste, aussi concise et complète que possible,
  d'axiomes formels auxquels ces opérations doivent obéir.  E.g. un de
  ces axiomes formalisera le fait qu'un \haskellinline{tmLookup} d'une clé
  $x$ juste après un \haskellinline{tmInsert} de la même clé avec une
  valeur $v$ devrait trouver $v$.
\item Donner le code des trois fonctions.
\end{enumerate}
Pour rendre l'exercice plus utile, il est important de faire ces étapes dans
l'ordre: i.e. ne pas écrire le code avant d'avoir décidé du type
des fonctions et de leurs spécifications.
\end{Exercise}

\begin{Answer}[ref={ex:treemap}]
  \haskellinput{exercices/treemap_Haskell_solution.hs}
\end{Answer}

\begin{Exercise}[title={Calculatrice en Haskell --  2}]
  \label{ex:calculatrice2}
Vous devez écrire le vérificateur de types et l'évaluateur pour un
langage uniquement composé d'opérateur arithmétique. Les opérateurs
disponibles sont \codeinline{+}, \codeinline{-}, \codeinline{==},
\codeinline{<}, \codeinline{if} et \codeinline{not}. Ils peuvent être
encodés par le datatype ci-dessous : \haskellinput[firstline=6,
lastline=12]{exercices/calculatrice_Haskell2_solution.hs}

Le langage est composé d'applications préfixes d'opérateurs, de nombres
entiers et de booléens. Le datatype \codeinline{Exp} ci-dessous peut
être utilisé comme arbre de syntaxe abstraite.  \haskellinput[firstline=17,
lastline=21]{exercices/calculatrice_Haskell2_solution.hs}
Par exemple, l'ASA ci-dessous est équivalent à 
\codeinline{(if (< 1 2) True False)}
\begin{minted}{text}
EApp [EOp OIf, (EApp [EOp OLessThan, EInt 1, EInt 2]), 
               EBool True, 
               EBool False]
\end{minted}

Le résultat d'une évaluation est une valeur représentée par le type
\codeinline{Value}. \haskellinput[firstline=28,
lastline=30]{exercices/calculatrice_Haskell2_solution.hs}

Vous devez écrire la fonction typeCheck qui doit vérifier que les
expressions sont valides. Chaque expression peut uniquement être de
type entier ou booléen.
\haskellinput[firstline=32,lastline=34]{exercices/calculatrice_Haskell2_solution.hs} Les
applications partielles d'opérateur ne sont pas permises. Évidemment
une application avec trop de paramètres est également une erreur. Les
opérateurs acceptent soit des entiers ou des booléens selon leur
sémantique habituelle. La branche vraie et la branche alternative du
\codeinline{if} doivent être du même type. Votre fonction
\codeinline{typeCheck} doit avoir la signature suivante:
\haskellinput[firstline=36,lastline=36]{exercices/calculatrice_Haskell2_solution.hs}

Une fois la fonction \codeinline{typeCheck} écrite, vous pouvez écrire
la fonction \codeinline{eval} sans avoir à gérer les erreurs. Ainsi,
les cas impossibles lorsqu'une expression est bien typée, une
application de \codeinline{+} avec un seul argument par exemple, n'ont
pas à être gérés. Votre fonction \codeinline{eval} aura pour signature:
\haskellinput[firstline=83,lastline=83]{exercices/calculatrice_Haskell2_solution.hs}
\end{Exercise}

\begin{Answer}[ref={ex:calculatrice2}]
  \haskellinput{exercices/calculatrice_Haskell2_solution.hs}
\end{Answer}

\begin{Exercise}[title={Vers l'infini et plus loin encore !}]
Les listes infinies sont souvent appelées \emph{streams}.  En Haskell,
l'ordre d'évaluation utilisé permet d'utiliser n'importe quelle structure de
donnée infinie sans effort particulier.  Prenons par exemple les définitions
ci-dessous:
\begin{minted}{Haskell}
zeros = 0 : zeros
uns = 1 : uns
\end{minted}

De plus, Haskell prédéfini les opérations suivantes:
\begin{minted}{Haskell}
(x:_)  !! 0 = x
(_:xs) !! n = xs !! (n - 1)

take 0 _  = []
take _ [] = []
take n (x:xs) = x : take (n - 1) xs

zipWith :: (a->b->c) -> [a]->[b]->[c]
zipWith _ [] _ = []
zipWith _ _ [] = []
zipWith f  (a:as) (b:bs) = f a b : zipWith f as bs
\end{minted}

\begin{enumerate}
\item Définir  la liste de nombres :
\begin{verbatim}
  (1 2 3 4 5 ...
\end{verbatim}
\item Définir  la liste des nombres de Fibonacci :
\begin{verbatim}
  (1 1 2 3 5 8 13 ...
\end{verbatim}
\item Définir  la liste de nombres :
\begin{verbatim}
  (1 1/2 1/6 1/24 1/120 ... 1/n! ...
\end{verbatim}
\end{enumerate}
\end{Exercise}

\input{exercices/programmation_fonctionnelle_C.tex}

\chapter{Gestion mémoire}

\begin{Exercise}[title={Compteur de références}]
Soit une libraire de gestion de listes simplement chaînées en C:
\begin{minted}{C}
typedef struct list list;
struct list {
  int refcount;
  void *head;
  list *tail;
}
list *list_cons   (void *head, list *tail);
void *list_head    (list *l);
list *list_tail    (list *l);
/* Copie un pointeur (pas la liste elle même).  */
list *list_copy   (list *l);
/* Libère un pointeur (et la liste si c'est le dernier).  */
void  list_free   (list *l);
\end{minted}

\begin{enumerate}
\item Écrire le code des fonctions proposées.
\item Compléter en ajoutant une opération \codeinline{list_map}.
\item Justifiez pourquoi les incréments et décréments que vous avez
judicieusement placés sont suffisants pour garantir que le comportement
sera correct.
\item En extraire une convention spécifiant pour les programmeurs qui
utilisent votre librarie \emph{où} doivent être ajoutés les appels à
\codeinline{list_copy} et \codeinline{list_free}.
\item Que se passe-t-il si vous voulez manipuler des listes de listes?
\end{enumerate}
\end{Exercise}

\input{exercices/gestion_mémoire.tex}

\begin{Exercise}[title={Malloc, free et mmap}]
\begin{enumerate}
\item Que se passe-t-il si vous passez à la fonction \codeinline{free}
  un pointeur qui n'a jamais été retourné par la fonction
  \codeinline{malloc} ? Que se passe-t-il si vous passer un pointeur
  de valeur \codeinline{NULL} ?
\item Que fait la fonction \codeinline{mmap} qui se trouve dans la
  librairie \codeinline{<sys/mman.h>} sous Unix ?
\item Quelle est la différence entre \codeinline{mmap} et \codeinline{malloc} ?
\end{enumerate}
\end{Exercise}


\chapter{Continuation}
\begin{Exercise}
Réécrire les fonctions suivantes en mode CPS (continuation passing
style). C'est à dire que chaque fonction doit prendre une continuation
et envoyer son résultat à cette continuation.

\begin{minted}{Haskell}
import Prelude hiding (length)

length :: [a] -> Int
length [] = 0
length (_ : xs) = 1 + length xs
  
applatir :: [[a]] -> [a]
applatir [] = []
applatir (xs : xss) = xs ++ applatir xss

fact :: Int -> Int
fact 0 = 1
fact n = n * fact (n - 1)

fib :: Int -> Int
fib 0 = 0
fib 1 = 1
fib n = fib (n - 1) + fib (n - 2)
\end{minted}
\end{Exercise}

\chapter{Macros}
\begin{Exercise}[title={Macros et passage par nom}]
\label{ex:macro_appel_nom}
  \begin{enumerate}
\item Définir une macro qui démontre que les macros implémentent le
  passage par nom. Dans quel cas cela fait-il une différence par
  rapport au passage par valeur ?
\item Indiquer comment obtenir un passage par valeur.
\item Votre solution pour le passage par valeur a fort probablement
  introduit un nouveau problème, lequel et comment le régler ?
\end{enumerate}
\end{Exercise}
\begin{Answer}[ref={ex:macro_appel_nom}]
  \schemeinput{exercices/macro_appel_par_nom.rkt}
\end{Answer}


\begin{Exercise}[title={Macro case}]
Soit une macro \codeinline{case} en Scheme qui peut s'utiliser comme suit:
\begin{minted}{scheme}
  (case <exp>
    ((1 3 5) <exp1>)
    ((4) <exp2>)
    (else <exp3>))
\end{minted}
qui signifierait exécuter \codeinline{exp1} si \codeinline{exp} vaut 1, 3, ou 5;
exécuter \codeinline{exp2} si \codeinline{exp} vaut 4 et exécuter \codeinline{exp3} sinon.

\begin{itemize}
\item Écrire une définition (naïve) de cette macro en Scheme avec
  \codeinline{define-macro}.

\item Discuter des différents problèmes qui peuvent apparaître lors de
  l'usage de cette macro dûs à son implémantation naïve.  Montrer des
  exemples concrets d'usage où la macro ne fait pas ce que le
  programmeur attendait.

\item Écrire une implémentation plus rafinée qui fait très attention à ce
  que le code donne une sémantique propre, sans mauvaises surprises pour
  l'utilisateur de cette macro.
\end{itemize}
\end{Exercise}

\begin{Exercise}[title={Macro postfix}]
\label{ex:macro_postfix}
  Définir la macro \codeinline{postfix} qui prend une expression sous forme
postfixée:

\begin{minted}{scheme}
  (let ((x 5))
    (postfix 1 x + 3 * 2 /)) ; retourne 9
\end{minted}

\noindent
Il suffira d'accepter les opérateurs \codeinline{+}, \codeinline{-}, \codeinline{*}, \codeinline{/},
\codeinline{not}, $\geq$, et \codeinline{if}.

Montrer les étapes de la compilation et de l'évaluation de
\codeinline{(postfix 1 x + 3 * 2 /)} ci-dessus, jusqu'à l'obtention du résultat 9, en indiquant
clairement quelles parties ont lieu lors de la compilation et quelles parties
ont lieu à l'exécution.
\end{Exercise}

\begin{Answer}[ref={ex:macro_postfix}]
  \schemeinput{exercices/macro_postfixe.scm}
\end{Answer}

\begin{Exercise}[title={Macro infixe}]
Faire une macro pour la notation infixe des opérateurs \codeinline{+},
\codeinline{-},\codeinline{*}, \codeinline{/}, \codeinline{>},
\codeinline{<}, \codeinline{=}.

Par exemple :
\begin{minted}{text}
(let ((x 5))
  (infix (1 + x) * 3 / 2)) ; 9

(let ((x 5)
      (y 10))
  (infix x * 2 < y - 1)) ; #f

(let ((x 5)
      (y 10))
  (infix x * 2 = y )) ; #t
\end{minted}
\end{Exercise}

Avec la forme \codeinline{quasiquote} écrire une macro qui permet de faire des
boucles \codeinline{while} avec la syntaxe suivante.

\begin{minted}{scheme}
  (while <condition> <expression>)
\end{minted}

Utiliser \codeinline{gensym} pour tout nouvel identificateur introduit par la
macro pour éviter la capture de variable.  Voici un exemple
d'utilisation:

\begin{minted}{scheme}
     (let ((i 1))
       (while (< i 10)
         (begin
           (write i)
           (set! i (* i 2)))))
\end{minted}

\begin{Exercise}[title={Macro trace}]
À l'aide de define-macro, écrire la macro
\codeinline{define-avec-trace} qui a le comportement suivant :

\begin{minted}{scheme}
> (define-avec-trace f (lambda (n) (if (= n 0) 1 (* n (f (- n 1))))))
> (f 5)
  (fonction: f parametres: (n = 5))
  (fonction: f parametres: (n = 4))
  (fonction: f parametres: (n = 3))
  (fonction: f parametres: (n = 2))
  (fonction: f parametres: (n = 1))
  (fonction: f parametres: (n = 0))
  120
\end{minted}
\end{Exercise}

\begin{Exercise}[title={Macro let et let*}]
Après vous être renseigné sur la définition de \codeinline{let} et
\codeinline{let*} de Scheme, écrire une macro \codeinline{mylet} et
une macro \codeinline{mylet*} qui fournissent les mêmes
fonctionnalités et mais dont la définition n'utilisent que des fonctions anonymes
(\codeinline{(lambda ... ...)}) et l'application de fonction dans leur expansion.
\end{Exercise}

\chapter{Programmation logique}
\begin{Exercise}[title={Multiplication}]
Soit \codeinline{add(X,Y,Z)} une relation qui dit que \codeinline{Z}
est la somme de \codeinline{X} et \codeinline{Y}.  Définir la relation
\codeinline{mult(X,Y,Z)} qui fait la même chose pour la multiplication en
utilisant \codeinline{add}.
\end{Exercise}

\begin{Exercise}
La relation \codeinline{axe(X,Y,Z)} telle que définie ci-après peut donner des
solutions redondantes.
%%
\begin{minted}{prolog}
axe(_,0,0).
axe(0,_,0).
\end{minted}
%%
Donner d'abord un exemple d'interaction avec le système GNU Prolog qui
montre cette redondance.

Utiliser ensuite l'opérateur prédéfini \codeinline{T1 \== T2} pour
corriger la définition précédente de manière à éviter les solutions
redondantes produites par la relation \codeinline{axe(X,Y,Z)}.
L'opérateur \codeinline{T1 \== T2} permet de tester si les termes
\emph{clos} \codeinline et \codeinline{T2} sont différents.  C'est un opérateur
impur car il ne permet pas, par exemple, d'énumérer tous les termes
\codeinline{T2} possibles qui sont différents de \codeinline{T1}.
\end{Exercise}

\begin{Exercise}
Soit le code Prolog suivant qui interprète un mini langage typé
dynamiquement et composé seulement d'entiers \codeinline{int(N)}, de
variables \codeinline{id(X)}, de fonctions \codeinline{lambda(X,E)},
et d'applications de fonctions \codeinline{app(E1,E2)}:

\begin{minted}{prolog}
%% subst(+E1, +E2, +X, -V)
%% indique que la substitution de X par V dans E1 renvoie E2.
%% On présume que V est fermé!
subst(int(N), int(N), _, _).
subst(id(X), V, id(X), V).
subst(id(Y), id(Y), id(X), _) :- X \= Y.
subst(lambda(X, E), lambda(X, E), X, _).
subst(lambda(Y, Ea), lambda(Y, Eb), X, V) :-
    X \= Y, subst(Ea, Eb, X, V).
subst(app(E1a, E2a), app(E1b, E2b), X, V) :-
    subst(E1a, E1b, X, V), subst(E2a, E2b, X, V).

%% reduce(+E, -V)
%% indique que l'évaluation de E renvoie V.
reduce(int(N), int(N)).
reduce(lambda(X, B), lambda(X, B))
reduce(app(E1, E2), V) :-
    reduce(E1, lambda(X, B)),
    reduce(E2, V2),
    subst(B, E, X, V2),
    reduce(E, V).
\end{minted}

\begin{enumerate}

\item Quel mode de passage d'arguments l'interpéteur ci-dessus implante-t-il ?
\item Modifier le code Prolog de sorte à implanter l'autre mode de
  passage d'arguments.
\item Est-ce que la portée des variables est statique ou dynamique ?
  Donner un morceau de code prolog qui fait la différence.

\end{enumerate}
\end{Exercise}

\begin{Exercise}
Supposons qu'on a déjà défini les relations suivantes:\medskip\\
%%
\begin{tabular}{l@{\hspace{1cm}}l}
  \codeinline{pere(X,Y)}  & \codeinline{X} est le père de \codeinline{Y} \\
  \verb|mere(X,Y)|  & \codeinline{X} est la mère de \codeinline{Y}
\end{tabular} \medskip\\
%%
Définissez les relations suivantes, en évitant autant que
possible les solutions redondantes et les boucles infinies.
Vous pouvez supposer qu'il n'y a pas d'enfants issus d'un
couple consanguin. \medskip\\
%%
\begin{tabular}{l@{\hspace{1cm}}l}
  \verb|parent(X,Y)|		& \codeinline{X} est un parent de \codeinline{Y} \\
  \verb|grandpere(X,Y)|	& \codeinline{X} est un grand-père de \codeinline{Y} \\
  \verb|grandmere(X,Y)|	& \codeinline{X} est une grand-mère de \codeinline{Y} \\
  \verb|freresoeur(X,Y)|	& \codeinline{X} est un frère ou une soeur de \codeinline{Y} \\
  \verb|oncletante(X,Y)|	& \codeinline{X} est un oncle ou une tante de \codeinline{Y}
\end{tabular} \medskip\\
\end{Exercise}

\begin{Exercise}
Soit la relation \codeinline{membre} qui défini quand un élément est dans une liste:
%%
\begin{minted}{prolog}
  membre(E, [E | _]).
  membre(E, [_ | L]) :- membre(E, L).
\end{minted}
%%
\begin{enumerate}
\item Montrer l'arbre de preuve des étapes importantes par lesquelles passe
  le système Prolog pour essayer de satisfaire la requête:
  \begin{displaymath}
    \verb|membre(X, [1, 1]), X == 2.|
  \end{displaymath}
  Où \texttt{A == B} est le prédicat qui demande l'unification de $A$ et de
  $B$. \\
  Les étapes importantes sont celles où la recherche tombe sur une
  impossibilité (et doit donc faire un retour-arrière).%%   Utiliser la
  %% notation $\frac{\textsl{premisses}}{\textsl{conclusion}}$ pour représenter
  %% les arbres de preuve.
\item Corriger la relation avec l'aide de l'inégalité \verb|\==| de sorte
  qu'elle évite la redondance.
\item En utilisant votre nouvelle définition, dessiner l'arbre de preuve
  qui satisfait la requète \verb|membre(1,[2,1,3]).|
\item De même montrer les arbres de preuve importants construits pour essayer
  de satisfaire la requête \verb|membre(1,[2,3,4]).|
\end{enumerate}
\end{Exercise}

\begin{Exercise}
Définir la règle de tri \verb|quicksort(X,Y)| qui dit que \codeinline{Y} contient
les mêmes éléments que la liste \codeinline{X}, mais triés par ordre croissant.
Cette version utilisera un opérateur de comparaison fixe, la relation
\textsl{`}$<$\textsl{'}.

Une régle auxiliaire \verb|partition(X,L,S,G)| sera nécessaire qui dit
que la liste \codeinline{S} contient les éléments de la liste \codeinline{L} qui sont plus
petits que \codeinline{X}, alors que \codeinline{G} contient ceux qui sont plus grand.

Utiliser la règle prédéfinie \verb|append(X,Y,Z)| qui dit que \codeinline{Z} est la
concaténation des listes \codeinline{X} et \codeinline{Y}.
\end{Exercise}

\begin{Exercise}
  \label{ex:prolog_regex}
En Prolog, tout comme en Haskell, les chaînes de caractères sont
représentées par des \emph{listes} de caractères.

Soit la définition suivante du prédicat de filtrage
\verb+match(RE,Str,Tail)+ qui dit que l'expression régulière \texttt{RE}
filtre la chaîne de caractères \texttt{Str} avec un résidu \texttt{Tail}.

\begin{minted}{prolog}
  match(RE, Str, Tail) :- append(RE, Tail, Str).
  match(any, [_|Tail], Tail).
  match(concat(RE1, RE2), Str, Tail) :-
     match(RE1, Str, Tail1), match(RE2, Tail1, Tail).
  match(repeat(RE), Str, Tail) :-
     match(RE, Str, Tail1), match(repeat(RE), Tail1, Tail).
  match(repeat(_), Str, Str).
\end{minted}
%%
Par exemple:
\begin{minted}{prolog}
  | ?- match("hello", "hello world", X).
  X = " world"
  | ?- match(concat("hel", "lo "), X, "").
  X = "hello "
  | ?- match(concat(repeat(any), concat("or", repeat(any))),
             "hello world", X).
  X = ""
  X = "d"
  X = "ld"
\end{minted}

\begin{enumerate}
\item Montrer le (les) arbres de recherche de la requête: \\
  \verb+match(concat(repeat("a"),"ab"),"abc",X)+
\item Que se passe-t-il avec la requête: \\
  \verb+match(concat(repeat("a"), "ab"), X, "")+
\item Changer le code de manière à éviter ce problème.
\item Ajouter du code pour l'expression régulière \texttt{or(RE1,RE2)}, de
  sorte à pouvoir faire des choses telles que:
\begin{minted}{prolog}
  | ?- match(or("hello", "world"), X, "").
  X = "hello"
  X = "world"
  | ?- match(repeat(or("a", "b")), "abac", X).
  X = "c"
  X = "ac"
  X = "bac"
  X = "abac"
\end{minted}
\item Montrer le (les) arbres de recherche de la requête: \\
  \verb+match(repeat(or("a", "b")), "abac", X)+
\item Ajouter du code pour l'expression régulière \texttt{and(RE1,RE2)}, de
  sorte à pouvoir faire des choses telles que:
\begin{minted}{prolog}
  | ?- match(concat(and(concat(repeat(any),
                               concat("ol",repeat(any))),
                        concat(repeat(any),
                               concat("ll",repeat(any)))),
                    "o"),
             "bell hollow world", X).
  X = "rld"
  X = "rld"
  X = "w world"
  X = "w world"
\end{minted}

\end{enumerate}
\end{Exercise}

\begin{Answer}[ref={ex:prolog_regex}]
  \schemeinput{exercices/prolog_regex.pl}
\end{Answer}

\chapter{Solutions}
\shipoutAnswer

\end{document}