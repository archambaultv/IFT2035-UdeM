\begin{Exercise}
\label{ex:prolog_quicksort}
Définir la règle de tri \verb|quicksort(X,Y)| qui dit que \codeinline{Y} contient
les mêmes éléments que la liste \codeinline{X}, mais triés par ordre croissant.
Cette version utilisera un opérateur de comparaison fixe, la relation
\textsl{`}$<$\textsl{'}.

Une régle auxiliaire \verb|partition(X,L,S,G)| sera nécessaire qui dit
que la liste \codeinline{S} contient les éléments de la liste \codeinline{L} qui sont plus
petits que \codeinline{X}, alors que \codeinline{G} contient ceux qui sont plus grand.

Utiliser la règle prédéfinie \verb|append(X,Y,Z)| qui dit que \codeinline{Z} est la
concaténation des listes \codeinline{X} et \codeinline{Y}.
\end{Exercise}

\begin{Answer}[ref={ex:prolog_quicksort}]
  \schemeinput{exercices/prolog_quicksort.pl}
\end{Answer}