\begin{Exercise}[title={Calculatrice Haskell}]
Écrire une petite calculatrice Haskell qui fait des additions.
Le langage arithmétique de cette calculatrice correspond au datatype suivant:
\begin{minted}{Haskell}
data Exp = Enum Int
         | Eplus Exp Exp
\end{minted}

Il faut écrire la fonction eval qui convertit ces expressions en nombre. Cette fonction à la signature suivante :
\begin{minted}{Haskell}
eval :: Exp -> Int
\end{minted}

Voici trois exemples :
\begin{itemize}
\item 
\begin{minted}{Haskell}
eval (Enum 1) == 1
\end{minted}

\item 
\begin{minted}{Haskell}
eval (Eplus (Enum 1) (Enum 1)) == 2
\end{minted}

\item 
\begin{minted}{Haskell}
eval (Eplus (Eplus (Enum 1) (Enum 1)) (Enum 1)) == 3
\end{minted}
\end{itemize}
\end{Exercise}