\begin{Exercise}[title={Récursion et arbre binaire}]
  \label{ex:recursion_arbre_binaire}
Soit un arbre binaire défini par le datatype ci-dessous
\begin{minted}{haskell}
data Tree = Leaf Int
          | Node Tree Int Tree
\end{minted}

\noindent Écrire le code des fonctions suivantes :
\begin{description}
\item[nbrLeaf] Retourne le nombre de feuilles dans l'arbre
\item[nbrNodes] Retourne le nombre de noeuds dans l'arbre (la fonction
  ne compte pas les feuilles)
\item[applyFunction] Prend une fonction $f$ de type \haskellinline{Int -> Int} 
  en paramètre ainsi qu'un arbre $t$ et applique la fonction
  $f$ à l'entier de chaque \haskellinline{Node} et \haskellinline{Leaf} de
  $t$.

  La signature de applyFunction est:

 \haskellinline{applyFunction :: (Int -> Int) -> Tree -> Tree}.
 
\end{description}
\end{Exercise}

\begin{Answer}[ref={ex:recursion_arbre_binaire}]
  \haskellinput{exercices/récursion2_solution.hs}
\end{Answer}