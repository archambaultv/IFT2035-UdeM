\begin{Exercise}
Soit le code Prolog suivant qui interprète un mini langage typé
dynamiquement et composé seulement d'entiers \codeinline{int(N)}, de
variables \codeinline{id(X)}, de fonctions \codeinline{lambda(X,E)},
et d'applications de fonctions \codeinline{app(E1,E2)}:

\begin{minted}{prolog}
%% subst(+E1, +E2, +X, -V)
%% indique que la substitution de X par V dans E1 renvoie E2.
%% On présume que V est fermé!
subst(int(N), int(N), _, _).
subst(id(X), V, id(X), V).
subst(id(Y), id(Y), id(X), _) :- X \= Y.
subst(lambda(X, E), lambda(X, E), X, _).
subst(lambda(Y, Ea), lambda(Y, Eb), X, V) :-
    X \= Y, subst(Ea, Eb, X, V).
subst(app(E1a, E2a), app(E1b, E2b), X, V) :-
    subst(E1a, E1b, X, V), subst(E2a, E2b, X, V).

%% reduce(+E, -V)
%% indique que l'évaluation de E renvoie V.
reduce(int(N), int(N)).
reduce(lambda(X, B), lambda(X, B))
reduce(app(E1, E2), V) :-
    reduce(E1, lambda(X, B)),
    reduce(E2, V2),
    subst(B, E, X, V2),
    reduce(E, V).
\end{minted}

\begin{enumerate}

\item Quel mode de passage d'arguments l'interpéteur ci-dessus implante-t-il ?
\item Modifier le code Prolog de sorte à implanter l'autre mode de
  passage d'arguments.
\item Est-ce que la portée des variables est statique ou dynamique ?
  Donner un morceau de code prolog qui fait la différence.

\end{enumerate}
\end{Exercise}