\begin{Exercise}[title={Calculatrice en Haskell --  2}]
  \label{ex:calculatrice2}
Vous devez écrire le vérificateur de types et l'évaluateur pour un
langage uniquement composé d'opérateur arithmétique. Les opérateurs
disponibles sont \codeinline{+}, \codeinline{-}, \codeinline{==},
\codeinline{<}, \codeinline{if} et \codeinline{not}. Ils peuvent être
encodés par le datatype ci-dessous : \haskellinput[firstline=6,
lastline=12]{exercices/calculatrice_Haskell2_solution.hs}

Le langage est composé d'applications préfixes d'opérateurs, de nombres
entiers et de booléens. Le datatype \codeinline{Exp} ci-dessous peut
être utilisé comme arbre de syntaxe abstraite.  \haskellinput[firstline=17,
lastline=21]{exercices/calculatrice_Haskell2_solution.hs}
Par exemple, l'ASA ci-dessous est équivalent à 
\codeinline{(if (< 1 2) True False)}
\begin{minted}{text}
EApp [EOp OIf, (EApp [EOp OLessThan, EInt 1, EInt 2]), 
               EBool True, 
               EBool False]
\end{minted}

Le résultat d'une évaluation est une valeur représentée par le type
\codeinline{Value}. \haskellinput[firstline=28,
lastline=30]{exercices/calculatrice_Haskell2_solution.hs}

Vous devez écrire la fonction typeCheck qui doit vérifier que les
expressions sont valides. Chaque expression peut uniquement être de
type entier ou booléen.
\haskellinput[firstline=32,lastline=34]{exercices/calculatrice_Haskell2_solution.hs} Les
applications partielles d'opérateur ne sont pas permises. Évidemment
une application avec trop de paramètres est également une erreur. Les
opérateurs acceptent soit des entiers ou des booléens selon leur
sémantique habituelle. La branche vraie et la branche alternative du
\codeinline{if} doivent être du même type. Votre fonction
\codeinline{typeCheck} doit avoir la signature suivante:
\haskellinput[firstline=36,lastline=36]{exercices/calculatrice_Haskell2_solution.hs}

Une fois la fonction \codeinline{typeCheck} écrite, vous pouvez écrire
la fonction \codeinline{eval} sans avoir à gérer les erreurs. Ainsi,
les cas impossibles lorsqu'une expression est bien typée, une
application de \codeinline{+} avec un seul argument par exemple, n'ont
pas à être gérés. Votre fonction \codeinline{eval} aura pour signature:
\haskellinput[firstline=83,lastline=83]{exercices/calculatrice_Haskell2_solution.hs}
\end{Exercise}

\begin{Answer}[ref={ex:calculatrice2}]
  \haskellinput{exercices/calculatrice_Haskell2_solution.hs}
\end{Answer}