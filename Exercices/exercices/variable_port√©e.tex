\begin{Exercise}
  \label{ex:portee_statique_dynamique}
Soit le code suivant dans un langage hypothétique dont la syntaxe est
la même que celle de Haskell: 
\begin{minted}{text}
  let z = 1
      x = 2
      f1 y = z + x + y
      f2 x = f1 (x + 1)
      f3 z = f3 (z + 2)
  in
      f3 5    
\end{minted}

Montrer les étapes de l'évaluation dans chacun des deux cas: le cas
où le langage utilise la portée dynamique et le cas où il utilise
la portée statique.
\end{Exercise}

\begin{Answer}[ref={ex:portee_statique_dynamique}]
  \codeinline{(x -> 5; y -> 6; ...) exp} dénote l'évaluation de \codeinline{exp}
  dans l'environnement où \codeinline{x} vaut 5, \codeinline{y} vaut 6, etc.  On
  peut imaginer l'environnement comme une pile. Lorsqu'on cherche la valeur d'un
  identificateur, on prend la première variable sur le dessus qui correspond.
  
  Pour rappel, \codeinline{\x -> body} est une fonction anonyme dont le
  paramètre est \codeinline{x}.

  \subsection*{Évaluation avec la portée dynamique}

\begin{minted}{text}
-- L'environnement avant d'évaluer (f3 5)
{f3 -> \z = f2 (z + 2);
 f2 -> \x = f1 (x + 1);
 f1 -> \y = z + x + y avec {z = 1; x = 2});
 x -> 2;
 z -> 1} (f3 5)

-- Recherche la valeur de f3 dans l'environnement
{...} ((\z = f2 (z + 2)) 5)

-- Application de fonction, on ajoute le
-- paramètre et sa valeur dans l'environnement
{z -> 5; ...} (f2 (z + 2))

-- Recherche la valeur de z dans l'environnement
{z -> 5; ...} (f2 (5 + 2))

-- Calcul du paramètre
{z -> 5; ...} (f2 7)

-- Recherche la valeur de f2 dans l'environnement
{z -> 5; ...} ((\x = f1 (x + 1)) 7)

-- Application de fonction
{x -> 7; z -> 5; ...} (f1 (x + 1))

-- Recherche la valeur de x dans l'environnement
{x -> 7; z -> 5; ...} (f1 (7 + 1)) 

-- Calcul du paramètre
{x -> 7; z -> 5; ...} (f1 8)

-- Recherche la valeur de f1 dans l'environnement
{x -> 7; z -> 5; ...} ((\y = z + x + y) 8)

-- Application de fonction
-- Ici x fait référence au x le plus récent
-- Même chose pour z
{y -> 8; x -> 7; z -> 5; ...} (z + x + y) z

-- Recherche la valeur de z, x et y
{y -> 8; x -> 7; z -> 5; ...} (5 + 7 + 8)

-- Calcul du résultat
{y -> 8; x -> 7; z -> 5; ...} 20
\end{minted}

\subsection*{Évaluation avec la portée statique}
Pour implanter la portée statique, il faut pour chaque expression mémoriser la
définition des variables libres. Ainsi, si d'autres variables avec le même
identifiant sont ajoutées dans l'environnement, cela n'affectera pas le résultat
car on utilisera la définition qui a été mémorisée.

Ainsi, on peut imaginer que chaque fonction a son propre environnement pour ses
variables libres. Pour simplifier, dans cet exercice nous mémoriserons seulement
la définition des variables \codeinline{x} et \codeinline{y}.


\begin{minted}{text}
-- L'environnement avant d'évaluer (f3 5)  
{f3 -> \z = f2 (z + 2);
 f2 -> \x = f1 (x + 1);
 f1 -> \y = z + x + y;
 x -> 2;
 z -> 1} (f3 5)

-- Recherche la valeur de f3 dans l'environnement
{...} ((\z = f2 (z + 2)) 5)

-- Application de fonction, on ajoute le
-- paramètre et sa valeur dans l'environnement
{z -> 5; ...} (f2 (z + 2))

-- Recherche la valeur de z dans l'environnement
{z -> 5; ...} (f2 (5 + 2))

-- Calcul du paramètre
{z -> 5; ...} (f2 7)

-- Recherche la valeur de f2 dans l'environnement
{z -> 5; ...} ((\x = f1 (x + 1)) 7)

-- Application de fonction
{x -> 7; z -> 5; ...} (f1 (x + 1))

-- Recherche la valeur de x dans l'environnement
{x -> 7; z -> 5; ...} (f1 (7 + 1)) 

-- Calcul du paramètre
{x -> 7; z -> 5; ...} (f1 8)

-- Recherche la valeur de f1 dans l'environnement
-- f1 a mémorisé les définitions pour x et z
-- Cela revient à dire que f1 s'évalue dans cet
-- environnement spécial
{x -> 2; z -> 1} ((\y = z + x + y) 8)

-- Application de fonction
{y -> 8; x -> 2; z -> 1} (z + x + y) z

-- Recherche la valeur de z, x et y
{y -> 8; x -> 2; z -> 1} (8 + 2 + 1)

-- Calcul du résultat
{y -> 8; x -> 2; z -> 1} 11
\end{minted}

\end{Answer}