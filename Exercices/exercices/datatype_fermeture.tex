\begin{Exercise}
\label{ex:datatype_fermeture}
Cet exercice a pour objectif de vous faire pratiquer les fermetures et de
vous convaincre qu'elles peuvent servir de structure de données. Nous
allons utiliser les encodages de Church pour écrire les structures de
données usuelles sous forme de fonctions. Chaque fonction peut être
écrite sur une ligne, mais ce sont des lignes qui ne sont pas toujours
facile à trouver.

\subsection*{Détail technique}
Nous allons redéfinir plusieurs identificateurs de la libraire
standard. Je vous conseille de mettre la ligne suivant au début de
votre fichier Haskell pour éviter que le compilateur ou ghci vous
indique que vous faites ombrage aux définitions standards.
\begin{minted}{haskell}
import Prelude hiding (Bool, succ, fst, snd, head, tail)
\end{minted}

\ExePart[title={Booléen}]
\label{sec:boolean}
Vous devez implanter les booléens sous forme de fonctions. Pour vous
donner un indice, le type des deux fonctions qui représentent le
boolen \codeinline{true} et \codeinline{false} est le suivant:
\begin{minted}{haskell}
type Bool a = a -> a -> a
\end{minted}

Vous devez :
\begin{enumerate}
\item Écrire la fonction \codeinline{true} et \codeinline{false} de
  type \codeinline{Bool a}. En fait, il n'y a que deux fonctions
  possibles avec ce type.
\item Écrire la fonction \codeinline{if2} qui a pour type
  \codeinline{Bool a -> a -> a -> a}. C'est à dire quelle prend un
  booléen et deux valeur de type \codeinline{a} et retourne la première
  valeur si le booléen est \codeinline{true} et la deuxième sinon.
\end{enumerate}

Par exemple, dans le code suivant \codeinline{testIfTrue} vaut 1
\begin{minted}{haskell}
testIfTrue = if2 true 1 2
\end{minted}
et dans le code suivant \codeinline{testIfFalse} vaut 2
\begin{minted}{haskell}
testIfFalse = if2 false 1 2
\end{minted}

\ExePart[title={Nombres naturels}]
\label{sec:nombre_naturel}
Vous devez implanter les nombres naturel $0 \ 1 \ 2 \ 3 \ldots$ à l'aide de
fonctions. Le type d'un nombre naturel est :
\begin{minted}{haskell}
type Number t = (t -> t) -> t -> t
\end{minted}

La logique de l'encodage est la suivante : un nombre est représenté
par une fonction qui attend une valeur de base de type \codeinline{t}
et une fonction de type \codeinline{t -> t}. Remarquez que la fonction
retourne une valeur de type \codeinline{t}. L'idée est que 0
correspond à la valeur de base, 1 correspond à \emph{une} application
de la fonction, 2 correspond à \emph{deux} applications de la fonction,
3 correspond à \emph{trois} applications de la fonction, etc.

Vous devez :
\begin{enumerate}
\item Écrire la fonction \codeinline{zero} de type \codeinline{Number
    t} qui correspond au nombre 0.
\item Écrire la fonction \codeinline{succ} de type 
    \codeinline{Number t -> Number t} qui retourne le successeur du nombre
    reçu en argument. Pour vous aider, voici une définition partielle de 
    \codeinline{succ}
\begin{minted}{haskell}
succ :: Number t -> Number t
succ n = \f z ->
\end{minted}
  \item Écrire la fonction \codeinline{plus} de type
    \codeinline{Number t -> Number t -> Number t} qui prend deux
    entiers et retourne un entier qui représente l'addition des deux.
\end{enumerate}

Par exemple, dans le code suivant \codeinline{myOne} vaut 1
\begin{minted}{haskell}
one :: Number t
one = succ zero

myOne = one (+ 1) 0
\end{minted}
et dans le code suivant \codeinline{myTwo} vaut 10
\begin{minted}{haskell}
two :: Number t
two = plus (succ zero) (succ zero)

myTwo = two (+ 5) 0
\end{minted}

\ExePart[title={Paires}]
\label{sec:paires}
Vous devez implanter le concept d'une paire à l'aide de fonctions. Le
type d'une paire est :
\begin{minted}{haskell}
type Pair a b t = (a ->  b ->  t) ->  t
\end{minted}

Vous devez :
\begin{enumerate}
\item Écrire la fonction \codeinline{mkPair} de type
  \codeinline{a -> b -> Pair a b t} qui construit une
  paire. Pour vous aider, voici une définition partielle de
  \codeinline{mkPair}
\begin{minted}{haskell}
mkPair :: a -> b -> Pair a b t
mkPair a b = \f -> 
\end{minted}

\item Écrire la fonction \codeinline{fst} de type \codeinline{Pair a b
    a -> a} qui retourne le premier élément de la paire.

\item Écrire la fonction \codeinline{snd} de type \codeinline{Pair a b
    b -> b} qui retourne le deuxième élément de la paire.
\end{enumerate}

Par exemple, dans le code suivant \codeinline{x} vaut 0
\begin{minted}{haskell}
x = fst (mkPair 0 "c")
\end{minted}
et dans le code suivant \codeinline{x} vaut \codeinline{"c"}
\begin{minted}{haskell}
x = snd (mkPair 0 "c")
\end{minted}

\ExePart[title={Liste}]
\label{sec:paires}
Finalement, vous devez implanter le concept d'une liste à l'aide de fonctions. Le
type d'une liste est :
\begin{minted}{haskell}
type List a t = (a->t->t)->t->t
\end{minted}
Ce type est très similaire à celui des entiers. Vous pouvez réutiliser
la même logique.

Vous devez :
\begin{enumerate}
\item Écrire la fonction \codeinline{cons} de type \codeinline{a ->
    List a t -> List a t}. Pour vous aider, voici une définition
  partielle de \codeinline{cons}
\begin{minted}{haskell}
cons :: a -> List a t -> List a t
cons x l = \f z ->
\end{minted}

\item Écrire la fonction \codeinline{nil} de type \codeinline{List a
    t} qui représente la liste vide.

\item Écrire la fonction \codeinline{isNil} de type \codeinline{List a
    (Bool t2) -> Bool t2} qui retourne vrai ou faux si la liste
  est vide ou non. Il s'agit bien du type \codeinline{Bool a} vu à la
  section \ref{sec:boolean}

\item Écrire la fonction \codeinline{head} de type \codeinline{List a
    a -> a -> a} qui retourne le premier élément de la liste. Le
  deuxième paramètre est une valeur par défaut qu'il faut retourner si
  la liste est vide.

\end{enumerate}

Par exemple, dans le code suivant \codeinline{x} vaut \codeinline{"vrai"}
\begin{minted}{haskell}
x =  isNil nil "vrai" "faux"
\end{minted}
et dans le code suivant \codeinline{x} vaut \codeinline{"faux"}
\begin{minted}{haskell}
x =  isNil (cons 1 nil) "vrai" "faux"
\end{minted}
et dans le code suivant \codeinline{x} vaut 1
\begin{minted}{haskell}
x =  head (cons 1 nil) 0
\end{minted}
et dans le code suivant \codeinline{x} vaut 0
\begin{minted}{haskell}
x =  head nil 0
\end{minted}

\subsection*{Pour les curieux}
Maintenant vous savez qu'en présence des fermetures il n'est pas
nécessaire (en théorie) d'avoir des structures de données. C'est
pourquoi le lambda calcul, qui est un langage minimaliste composé
uniquement de fonctions, est souvent utilisé pour modéliser les
langages fonctionnels.
\end{Exercise}

\begin{Answer}[ref={ex:datatype_fermeture}]
  \haskellinput{exercices/datatype_fermeture_solution.hs}
\end{Answer}