\begin{Exercise}[title={Macro postfix}]
Définir la macro \codeinline{postfix} qui prend une expression sous forme
postfixée:

\begin{minted}{scheme}
  (let ((x 5))
    (postfix 1 x + 3 * 2 /)) ; retourne 9
\end{minted}

\noindent
Il suffira d'accepter les opérateurs \codeinline{+}, \codeinline{-}, \codeinline{*}, \codeinline{/},
\codeinline{not}, $\geq$, et \codeinline{if}.

Montrer les étapes de la compilation et de l'évaluation de
\codeinline{(postfix 1 x + 3 * 2 /)} ci-dessus, jusqu'à l'obtention du résultat 9, en indiquant
clairement quelles parties ont lieu lors de la compilation et quelles parties
ont lieu à l'exécution.
\end{Exercise}