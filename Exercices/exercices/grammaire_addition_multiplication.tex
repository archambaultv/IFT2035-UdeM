\begin{Exercise}[title={Grammaire pour l'addition et la multiplication}]
  \label{ex:grammaire-add-mult}
Nous allons voir dans cette section plusieurs grammaires BNF
possibles pour l'addition et la multiplication. Vous verrez qu'il est
facile de générer une grammaire BNF ambiguë ou ne respectant pas la
priorité et l'associativité des opérations.

\ExePart[title={Grammaire 1}]
\label{expart:grammaire1}
Soit la grammaire ci-dessous:
\begin{align*}
  \text{<expr>}  &\text{::=  <expr> + <expr> | <chiffre>} \\
  \text{<chiffre>}  &\text{::=  0 | 1 | 2 | 3 | 4 | 5 | 6 | 7 | 8 | 9 }
\end{align*}
\begin{enumerate}
\item Montrer que dans cette grammaire, pour une même expression, il
  est possible de choisir une dérivation où l'opérateur est
  associatif à gauche et une dérivation ou l'opérateur est
  associatif à droite. Cela démontre que la grammaire est ambiguë.
\end{enumerate}
\vspace{-\bigskipamount}

\ExePart[title={Grammaire 2}]
Voici comment on pourrait transformer la grammaire 1 pour lever
l'ambiguïté.
\begin{align*}
  \text{<expr>}  &\text{::=  <expr> + <terme> | <chiffre>} \\
  \text{<terme>} &\text{::= '(' <expr> ')' | <chiffre>} \\
  \text{<chiffre>}  &\text{::=  0 | 1 | 2 | 3 | 4 | 5 | 6 | 7 | 8 | 9 }
\end{align*}
\begin{enumerate}
\item Montrer que cette grammaire est associative à gauche.
\item Donner l'arbre de dérivation de l'expression $1 + (1 + 1) + 1$.
\end{enumerate}
Noter que cette grammaire est récursive à gauche et associative à
gauche.

\ExePart[title={Grammaire 3}]
Donner une grammaire BNF similaire à la grammaire 1 mais cette
fois-ci associative à droite.

\ExePart[title={Grammaire 4}]
La grammaire 2 est associative à gauche et non ambiguë. Voici une
première tentative d'ajouter l'opérateur de multiplication.
\begin{align*}
  \text{<expr>}  &\text{::=  <expr> + <terme> |  <expr> * <terme> | <chiffre>} \\
  \text{<terme>} &\text{::= '(' <expr> ')' | <chiffre>} \\
  \text{<chiffre>}  &\text{::=  0 | 1 | 2 | 3 | 4 | 5 | 6 | 7 | 8 | 9 }
\end{align*}

\begin{enumerate}
\item Montrer que cette grammaire est non ambiguë.
\item Cette grammaire ne respecte pas la priorité des
  opérations. Donner l'arbre de dérivation des expressions
  $1 + 2 * 3$ et $1 * 2 + 3$.
\end{enumerate}
\vspace{-\bigskipamount}

\ExePart[title={Grammaire 5}]
Soit la grammaire ci-dessous qui corrige le défaut de la grammaire 4. 
\begin{align*}
  \text{<expr>}  &\text{::=  <expr> + <terme> |  <terme>} \\
  \text{<terme>} &\text{::=  <terme> * <facteur> | <facteur>} \\
  \text{<facteur>} &\text{::= '(' <expr> ')' | <chiffre>} \\
  \text{<chiffre>}  &\text{::=  0 | 1 | 2 | 3 | 4 | 5 | 6 | 7 | 8 | 9 }
\end{align*}

\begin{enumerate}
\item Cette grammaire respecte la priorité des opérations. Donner
  l'arbre de dérivation des expressions $1 + 2 * 3$ et $1 * 2 + 3$.
\end{enumerate}
\vspace{-\bigskipamount}

\ExePart[title={Grammaire 6}]
Voici comment ajouter la soustraction et la division. 
\begin{align*}
  \text{<expr>}  &\text{::=  <expr> + <terme> | <expr> - <terme> |  <terme>} \\
  \text{<terme>} &\text{::=  <terme> * <facteur> |  <terme> / <facteur> | <facteur>} \\
  \text{<facteur>} &\text{::= '(' <expr> ')' | <chiffre>} \\
  \text{<chiffre>}  &\text{::=  0 | 1 | 2 | 3 | 4 | 5 | 6 | 7 | 8 | 9 }
\end{align*}

\begin{enumerate}
\item Pourquoi peut-on ajouter la soustraction (division) dans la même
  catégorie que l'addition (multiplication) plutôt que de créer une
  nouvelle catégorie ?
\end{enumerate}
\vspace{-\bigskipamount}

\ExePart[title={Grammaire 7 - Pour les curieux}]
La grammaire 5 est non ambiguë et respecte la priorité et
l'associativité des opérateurs. Toutefois, elle est récursive à gauche
pour la catégorie <expr> et <terme>. Une grammaire récursive à gauche peut
engendrer des boucles infinies. 

En effet, dans un parseur avec analyse descendante (top down), la
portion du programme devant lire la catégorie <expr> doit d'abord faire
appel à la portion du programme qui doit lire la catégorie <expr> qui doit
d'abord faire appel à la portion du programme qui doit lire la catégorie
<expr> qui doit d'abord faire appel ...

Il faut donc transformer cette grammaire en une grammaire équivalente
mais non récursive à gauche.
\begin{align*}
  \text{<expr>}  &\text{::=  <terme> <expr'>} \\
  \text{<expr'>}  &\text{::=  + <terme> <expr'> | } \epsilon \\
  \text{<terme>} &\text{::=  <facteur> <terme'>} \\
  \text{<terme'>} &\text{::=  * <facteur> <terme'> | } \epsilon \\
  \text{<facteur>} &\text{::= '(' <expr> ')' | <chiffre>} \\
  \text{<chiffre>}  &\text{::=  0 | 1 | 2 | 3 | 4 | 5 | 6 | 7 | 8 | 9 }
\end{align*}

Il est possible de montrer que cette grammaire est équivalente à la
grammaire du numéro 5.
\end{Exercise}