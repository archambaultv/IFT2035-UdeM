\begin{Exercise}
  \label{ex:if_then_else}
Voici une grammaire pour if then else. Les règles <E> et <X>
représente des expressions et autres règles de la grammaire dont il
n'est pas important de spécifier.
\begin{minted}{text}
  <S> ::=  <X>
       | 'if' <E> 'then' <S>
       | 'if' <E> 'then' <S> 'else' <S>} 
\end{minted}

Cette grammaire est ambiguë. 
\begin{enumerate}
\item Donner un exemple d'ambiguïté.
\item Donner la grammaire non ambiguë qui associe les else avec le if
  le plus proche.
\end{enumerate}
\end{Exercise}

\begin{Answer}[ref={ex:if_then_else}]
  Voici une phrase ambigüe : \codeinline{if E1 then if E2 then E3 else E4}. Elle
  peut s'interpréter des deux façons ci-dessous.

  \begin{itemize}
  \item \codeinline{if E1 then (if E2 then E3 else E4)}
  \item \codeinline{if E1 then (if E2 then E3) else E4}.
  \end{itemize}
    
  Pour lever l'ambiguïté, il faut par exemple arbitrairement décider que les
  \codeinline{else} se réfèrent aux \codeinline{if} les plus proches.

\begin{minted}{text}
<S> ::= <X>
    | 'if' <E> 'then' <S>
    | 'if' <E> 'then' <SElse> 'else' <S>

<SElse> ::=
  | 'if' <E> 'then' <SElse> 'else' <SElse>
\end{minted}
\end{Answer}