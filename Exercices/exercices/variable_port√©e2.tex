\begin{Exercise}
  \label{ex:portee2}
\begin{enumerate}
\item Donner une définition de la portée statique (lexicale) et de la portée
dynamique.

\item 
Donner un exemple de code en utilisant la syntaxe Haskell qui sera
évalué différemment si la portée est lexicale ou dynamique. Expliquez
clairement pourquoi votre code se comporte différemment d'une portée à
l'autre.
\end{enumerate}
\end{Exercise}

\begin{Answer}[ref={ex:portee2}]
  \begin{enumerate}

  \item
    \begin{description}
    \item[Portée lexicale] La portée d'une variable est définie par le
      code source du programme.  La définition d'une variable est la
      définition dans le code source la plus récente ayant le même
      identificateur.
    \item [Portée dynamique] La portée d'une variable est définie lors
      de l'exécution du programme. La définition d'une variable est la
      définition la plus récemment évaluée (et toujours active) ayant
      le même identificateur.
    \end{description}

  \item
    \begin{minted}{text}
      x = 1
      foo y = y + x
      res = let x = 2
            in foo 0
    \end{minted}
    Avec la portée statique, la variable \codeinline{x} de la fonction
    foo vaut 1, car c'est la déclaration \codeinline{x = 1} qui est la
    plus proche dans le code source. Donc \codeinline{res} vaut 1.

    Avec la portée dynamique, la variable \codeinline{x} de la
    fonction foo vaut 2, car c'est la déclaration \codeinline{x = 2}
    du \codeinline{let} qui est la plus récemment executée et toujours
    active lors de l'évaluation du programme. Donc \codeinline{res}
    vaut 2.

  \end{enumerate}
\end{Answer}