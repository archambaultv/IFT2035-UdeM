\begin{Exercise}
\label{ex:postfixe}
Soit les expressions suivantes en notation postfixe:
\begin{enumerate}
\item \codeinline{a b + c - d *}
\item \codeinline{a b && c d == ||}
\item \label{exempleSyntaxe}\codeinline{a b + c d * == not}
\end{enumerate}

\begin{enumerate}
\item Écrire ces expressions en format infixe, préfixe et en arbre de
  syntaxe abstraite. \codeinline{==} est l'opérateur de comparaison,
  \codeinline{&&} est l'opérateur de conjonction et \codeinline{||}
  est l'opérateur de disjonction.


\item
Il est possible d'évaluer les expressions postfixes facilement à
l'aide d'une pile. Montrer comment évaluer l'expression numéro
\ref{exempleSyntaxe} de la page précédente avec une pile. Utiliser les
valeurs numériques suivantes : $a = 1$, $b = 2$, $c = 3$, $d = 4$.
\end{enumerate}
\end{Exercise}

\begin{Answer}[ref={ex:postfixe}]
  \begin{enumerate}
        \item
\begin{enumerate}[1.]
\item \begin{description}
  \item[préfixe]\codeinline{* - + a b c d}
  \item[infixe]\codeinline{(a + b - c) * d}
    \end{description}
\item \begin{description}
  \item[préfixe]\codeinline{|| && a b == c d}
  \item[infixe]\codeinline{(a && b) || (c == d)}
    \end{description}
\item \begin{description}
  \item[préfixe]\codeinline{not == + a b * c d}
  \item[infixe]\codeinline{not(a + b == c * d)}
    \end{description}
  \end{enumerate}

\item

  Il faut évaluer à l'aide d'une pile l'expression \codeinline{1 2 + 3 4 * == not}.
  La pile sera représenté par une liste,
  l'élément au somment de la pile est à gauche de la liste.

  Voici les étapes d'évaluation
  \begin{enumerate}[1.]
  \item Lecture du chiffre 1. Lorsque c'est un nombre, on l'ajoute dans la pile. Elle vaut \codeinline{[1]}
  \item Lecture du chiffre 2. La pile vaut \codeinline{[2, 1]}
  \item \label{ex:postfixe:itemplus} Lecture du symbole \codeinline{+}. On pop les deux premiers éléments de la pile, calcule le résultat et l'ajoute dans la pile. La pile vaut \codeinline{[3]}.
  \item Lecture du symbole 3. La pile vaut \codeinline{[3, 3]}
  \item Lecture du symbole 4. La pile vaut \codeinline{[4, 3, 3]}
  \item Lecture du symbole \codeinline{*}. Même chose que pour l'étape {\ref{ex:postfixe:itemplus}} La pile vaut \codeinline{[12, 3]}
  \item Lecture du symbole \codeinline{==}. Même chose que pour l'étape {\ref{ex:postfixe:itemplus}} La pile vaut \codeinline{[False]}
  \item Lecture du symbole \codeinline{not}. On pop le premier élément de la pile, calcule le résultat et l'ajoute sur la pile. La pile vaut \codeinline{[True]}
  \item Fin de l'expression. La valeur sur la pile donne le résultat, soit \codeinline{True}
  \end{enumerate}
  
  
\end{enumerate}
\end{Answer}