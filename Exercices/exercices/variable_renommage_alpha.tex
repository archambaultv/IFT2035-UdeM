\begin{Exercise}
\label{ex:variable_alpha}
Le code ci-dessous est écrit en Haskell et utilise donc la portée lexicale:
\begin{minted}{haskell}
x = 1
foo y = x + y
bar (x, y) = x + y
resultat = let bar x = x + 3
               x = 5
           in foo (bar x)
\end{minted}

\begin{enumerate}
\item Réécrire le code en renommant tous les identificateurs pour que chaque
variable ait un nom différent des autres. Bien sûr ce renommage ne
doit pas changer la sémantique du code.


\item Quelle est la valeur de \codeinline{resultat} ?
\item Quelle serait la valeur de \codeinline{resultat} si Haskell
  avait la portée dynamique ?
\end{enumerate}
\end{Exercise}

\begin{Answer}[ref={ex:variable_alpha}]
  \begin{enumerate}
  \item
    \begin{minted}{haskell}
x1 = 1
foo y1 = x1 + y1
bar1 (x2, y2) = x2 + y2
resultat = let bar2 x3 = x3 + 3
               x4 = 5
           in foo (bar2 x4)
         \end{minted}
       \item 9
         
       \item 13

         
    \end{enumerate}
\end{Answer}