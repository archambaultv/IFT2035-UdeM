\begin{Exercise}
\label{ex:fonction_raisonner2}
  Soit le morceau de code suivant, où \codeinline{f} est une fonction
quelconque que l'on ne connaît pas et où le langage utilise une
syntaxe de style C :
\begin{minted}{C}
void foo (void (*f) (char*, int)) {
 int x = 3, y = 2, z = 1;
 char *s1 = "haskell";
 char *s2 = strcpy (s1); /* fait une copie de s1 */
 while (--x > 0) {
  f (s2, y);
 }
 ...
}
\end{minted}
On aimerait savoir (par exemple pour décider de la validité de
certaines optimisations) si certaines conditions sont nécessairement
toujours vraies à la fin l'exécution du while.  On s'intéresse plus
particulièrement aux 5 conditions suivantes:
\begin{enumerate}
\item \codeinline{s2[5] == 'l'}
\item \codeinline{s1[1] == 'a'}
\item \codeinline{x == 0}
\item \codeinline{z == 1}
\item \codeinline{y == 2}
\end{enumerate}
Dans chacun des cas suivants, indiquer lesquelles de ces 5 conditions
sont nécessairement vraies et si non, justifier pourquoi pas. Vous
pouvez considérer que \codeinline{strcpy} fait uniquement une copie en mémoire et
rien d'autre.
\begin{enumerate}
\item Le langage est exactement comme C: portée statique, passage d'arguments
  par valeur et affectation autorisée.
\item Le langage est comme C sauf que tous les arguments sont passés
  par référence.
\item Le langage est comme C mais avec portée dynamique.
\end{enumerate}
\end{Exercise}

\begin{Answer}[ref={ex:fonction_raisonner2}]
  \begin{enumerate}
  \item \label{enum_raisonner2}
\begin{description}
\item[\codeinline{s2[5] == 'l'}]Faux, \codeinline{s2} est un pointeur et donc \codeinline{f} peut modifier le tableau
\item[\codeinline{s1[1] == 'a'}] Vrai
\item[\codeinline{x == 0}] Vrai
\item[\codeinline{z == 1}] Vrai
\item[\codeinline{y == 2}] Vrai
\end{description}

     \item
\begin{description}
\item[\codeinline{s2[5] == 'l'}] Faux, voir le point \ref{enum_raisonner2}
\item[\codeinline{s1[1] == 'a'}] Vrai
\item[\codeinline{x == 0}] Vrai
\item[\codeinline{z == 1}] Vrai
\item[\codeinline{y == 2}] Faux, \codeinline{f} reçoit y par référence et peut donc en modifier la valeur
\end{description}

     \item
\begin{description}
\item[\codeinline{s2[5] == 'l'}] Faux, \codeinline{s2} est accessible depuis \codeinline{f}.
\item[\codeinline{s1[1] == 'a'}] Faux, \codeinline{s1} est accessible depuis \codeinline{f}.
\item[\codeinline{x == 0}] Faux, \codeinline{x} est accessible depuis \codeinline{f}.
\item[\codeinline{z == 1}] Faux, \codeinline{z} est accessible depuis \codeinline{f}.
\item[\codeinline{y == 2}] Faux, \codeinline{y} est accessible depuis \codeinline{f}.
\end{description}
\end{enumerate}

\end{Answer}