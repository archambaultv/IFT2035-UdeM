\begin{Exercise}[title={Programmation fonctionnelle en C}]
Soit la fonction C suivante:
\begin{minted}{C}
void main (void)
{ /* Élimine les caractères répétés et
   * stoppe après la première ligne vide.  */
  int c;
  int last = EOF;
  while ((c = getchar ()) != EOF) {
    if (c == last) {
      if (c == '\n') {
        break;
      } else {
        continue;
      }
    }
    putchar (last = c);
  }
}
\end{minted}
D'abord, réécrire le code dans un style de programmation structurée
stricte, c'est à dire sans utiliser de \codeinline{continue},
\codeinline{break}, ou \codeinline{goto}.

Ensuite, réécrire le code à nouveau mais cette fois dans un style
fonctionnel, c'est à dire sans opération d'affectation (sauf bien sûr dans
les initialisations, e.g. \codeinline{int last = EOF}).  Il faudra
introduire une fonction récursive auxiliaire et éliminer \codeinline{while},
\codeinline{break}, et \codeinline{continue}.
\end{Exercise}