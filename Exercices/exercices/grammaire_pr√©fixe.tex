\begin{Exercise}
\label{ex:grammaire_prefixe}
Soit les expressions suivantes en notation préfixe:
\begin{enumerate}
\item \codeinline{/ - + a b c 2}
\item \codeinline{== + 1 x sqrt - * b b a}
\item \codeinline{- - a b c}
\item \codeinline{not == a - + a b sqrt c}
\item \codeinline{+ a + b + c + d + e + f g}
\end{enumerate}

Écrire ces expressions en format infixe, postfixe et en arbre de
syntaxe abstraite. \codeinline{==} est l'opérateur de comparaison,
\codeinline{sqrt} et \codeinline{not} sont des fonctions qui acceptent
un seul argument.
\end{Exercise}

\begin{Answer}[ref={ex:grammaire_prefixe}]

\begin{enumerate}
\item \begin{description}
  \item[infixe] \codeinline{(a + b - c) / 2}
  \item[postfixe] \codeinline{a b + c - 2 /}
  \end{description}
\item \begin{description}
  \item[infixe] \codeinline{1 + x == sqrt(b * b - a)}
  \item[postfixe] \codeinline{1 x + b b * a - sqrt ==}
  \end{description}
\item \begin{description}
  \item[infixe] \codeinline{a - b - c}
  \item[postfixe] \codeinline{a b - c -}
    \end{description}
  \item \begin{description}
    \item[infixe] \codeinline{not(a == a + b - sqrt(c))}
    \item[postfixe] \codeinline{a a b + c sqrt - == not}
    \end{description}
  \item \begin{description}
    \item[infixe] \codeinline{a + (b + (c + (d + (e + (f + g)))))}
    \item[postfixe] \codeinline{a b c d e f g + + + + + +}
      \end{description}
\end{enumerate}

\end{Answer}