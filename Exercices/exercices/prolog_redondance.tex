\begin{Exercise}
La relation \codeinline{axe(X,Y,Z)} telle que définie ci-après peut donner des
solutions redondantes.
%%
\begin{minted}{prolog}
axe(_,0,0).
axe(0,_,0).
\end{minted}
%%
Donner d'abord un exemple d'interaction avec le système GNU Prolog qui
montre cette redondance.

Utiliser ensuite l'opérateur prédéfini \codeinline{T1 \== T2} pour
corriger la définition précédente de manière à éviter les solutions
redondantes produites par la relation \codeinline{axe(X,Y,Z)}.
L'opérateur \codeinline{T1 \== T2} permet de tester si les termes
\emph{clos} \codeinline et \codeinline{T2} sont différents.  C'est un opérateur
impur car il ne permet pas, par exemple, d'énumérer tous les termes
\codeinline{T2} possibles qui sont différents de \codeinline{T1}.
\end{Exercise}