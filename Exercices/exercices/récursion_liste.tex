\begin{Exercise}[title={Récursion et liste}]
  \label{ex:recursion_liste}
  Écrire en Haskell les fonctions suivantes manipulant des listes de nombres  :
\begin{description}
\item[length] Retourne la longueur d'une liste. Exemples:
  \begin{itemize}
  \item \haskellinline[gobble=1]{ length [] == 0}
  \item \haskellinline[gobble=1]{ length [1] == 1}
  \item \haskellinline[gobble=1]{ length [1, 2] == 2}
  \end{itemize}
\item[concat] Retourne la concaténation de deux listes. Exemples 
  \begin{itemize}
  \item \haskellinline[gobble=1]{ concat [] [] == []}
  \item \haskellinline[gobble=1]{ concat [1] [2] == [1, 2]}
  \item \haskellinline[gobble=1]{ concat [] [2] == [2]}
  \item \haskellinline[gobble=1]{ concat [1, 2] [] == [1, 2]}
  \end{itemize}
\item[member] Trouve un nombre dans la liste. Exemples 
  \begin{itemize}
  \item \haskellinline[gobble=1]{ member 1 [] == False}
  \item \haskellinline[gobble=1]{ member 1 [2, 3, 1] == True}
  \end{itemize}
\item[reverse] Inverse la liste reçue en argument 
  \begin{itemize}
  \item \haskellinline[gobble=1]{ reverse [5, 4, 3, 2, 1] == [1, 2, 3, 4, 5]}
  \item \haskellinline[gobble=1]{ reverse [] == []}
  \end{itemize}
\item[subList] Indique si la première liste est incluse dans la
  seconde. Les éléments de la première liste doivent apparaître dans
  l'ordre dans la seconde. Exemples:
  \begin{itemize}
  \item \haskellinline[gobble=1]{ subList [] [5, 4, 3, 2, 1] == True}
  \item \haskellinline[gobble=1]{ subList [2, 1, 3] [5, 4, 3, 2, 1] == False}
  \item \haskellinline[gobble=1]{ subList [2, 1, 3] [5, 2, 1, 3, 4] == True}
  \item \haskellinline[gobble=1]{ subList [2] [] == False}
  \end{itemize}
\end{description}
\end{Exercise}

\begin{Answer}[ref={ex:recursion_liste}]
  \haskellinput{exercices/récursion_liste.hs}
\end{Answer}