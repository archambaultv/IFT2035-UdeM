\begin{Exercise}[title={Quicksort en Haskell}]
  \label{ex:quicksort_haskell}
Implanter en Haskell une variante de quicksort pour des listes
d'entiers.  En clair, trier une liste comme suit:
\begin{enumerate}
\item choisir un élément, que l'on nommera le pivot.
\item partitionner la liste en deux sous-listes d'éléments plus
  petits et respectivement plus grands que le pivot.
\item trier les deux sous-listes.
\item combiner ces sous-listes triées et le pivot en une liste triée.
\end{enumerate}
Le type sera: \haskellinline{quicksort :: [Int] -> [Int]}. Il faudra
peut-être définir une ou plusieurs fonctions auxiliaires. L'opération
de concaténation de deux listes s'écrit \haskellinline{++} en Haskell.

Finalement, généraliser la fonction de tri précédente pour pouvoir
l'appliquer à des listes quelconques (pas seulement
\haskellinline{Int}), en passant un argument supplémentaire qui
indique l'opération de comparaison à utiliser.

Donner aussi le type de cette fonction plus générale et de toutes les
fonctions auxiliaires que vous avez définies.
\end{Exercise}

\begin{Answer}[ref={ex:quicksort_haskell}]
  \haskellinput{exercices/pfonc_quicksort_Haskell.hs}
\end{Answer}