\begin{Exercise}
Soit la relation \codeinline{membre} qui défini quand un élément est dans une liste:
%%
\begin{minted}{prolog}
  membre(E, [E | _]).
  membre(E, [_ | L]) :- membre(E, L).
\end{minted}
%%
\begin{enumerate}
\item Montrer l'arbre de preuve des étapes importantes par lesquelles passe
  le système Prolog pour essayer de satisfaire la requête:
  \begin{displaymath}
    \verb|membre(X, [1, 1]), X == 2.|
  \end{displaymath}
  Où \texttt{A == B} est le prédicat qui demande l'unification de $A$ et de
  $B$. \\
  Les étapes importantes sont celles où la recherche tombe sur une
  impossibilité (et doit donc faire un retour-arrière).%%   Utiliser la
  %% notation $\frac{\textsl{premisses}}{\textsl{conclusion}}$ pour représenter
  %% les arbres de preuve.
\item Corriger la relation avec l'aide de l'inégalité \verb|\==| de sorte
  qu'elle évite la redondance.
\item En utilisant votre nouvelle définition, dessiner l'arbre de preuve
  qui satisfait la requète \verb|membre(1,[2,1,3]).|
\item De même montrer les arbres de preuve importants construits pour essayer
  de satisfaire la requête \verb|membre(1,[2,3,4]).|
\end{enumerate}
\end{Exercise}