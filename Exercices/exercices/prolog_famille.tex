\begin{Exercise}
Supposons qu'on a déjà défini les relations suivantes:\medskip\\
%%
\begin{tabular}{l@{\hspace{1cm}}l}
  \codeinline{pere(X,Y)}  & \codeinline{X} est le père de \codeinline{Y} \\
  \verb|mere(X,Y)|  & \codeinline{X} est la mère de \codeinline{Y}
\end{tabular} \medskip\\
%%
Définissez les relations suivantes, en évitant autant que
possible les solutions redondantes et les boucles infinies.
Vous pouvez supposer qu'il n'y a pas d'enfants issus d'un
couple consanguin. \medskip\\
%%
\begin{tabular}{l@{\hspace{1cm}}l}
  \verb|parent(X,Y)|		& \codeinline{X} est un parent de \codeinline{Y} \\
  \verb|grandpere(X,Y)|	& \codeinline{X} est un grand-père de \codeinline{Y} \\
  \verb|grandmere(X,Y)|	& \codeinline{X} est une grand-mère de \codeinline{Y} \\
  \verb|freresoeur(X,Y)|	& \codeinline{X} est un frère ou une soeur de \codeinline{Y} \\
  \verb|oncletante(X,Y)|	& \codeinline{X} est un oncle ou une tante de \codeinline{Y}
\end{tabular} \medskip\\
\end{Exercise}