\begin{Exercise}
  \label{ex:superhero}
  Vous devez aider un ami à traduire dans le langage C un programme
  écrit en Haskell.

\begin{enumerate}
\item
Convertir les datatypes Haskell ci-dessous dans le langage C:
\begin{minted}{Haskell}
data Personne = Personne Int String -- Age et Prénom

data SuperPouvoir = SauterHaut Int
                  | SauterLoin Int
                  | GrimperAuMur Int
                  | Imiter Personne

data SuperHero = SuperHero Personne SuperPouvoir

\end{minted}

\item
Écrire les fonctions Haskell ci-dessous dans le langage C dans un style
de programmation structurée (pas de goto).

Vous pouvez considérer que les listes Haskell sont des tableaux en C
et que la taille du tableau vous est fournie en paramètre.
\begin{minted}{Haskell}
plusVieux :: [SuperHero] -> [SuperHero]
plusVieux ls = 
  map (\(SuperHero (Personne age nom) sp) ->
         SuperHero (Personne (age + 1) nom) sp)
      ls

auMoinsUnSuperHero :: [SuperHero] -> Bool
auMoinsUnSuperHero ls =
  let foo (SuperHero (Personne _ "Spiderman") _) _ = True
      foo (SuperHero (Personne _ "Batman") _) _ = True
      foo (SuperHero (Personne _ "Capitaine Bobette") _) _ = True
      foo _ b = b
  in foldr foo False ls
\end{minted}

La fonction \codeinline{foldr} a pour définition :
\begin{minted}{text}
foldr f z [] = z 
foldr f z (x:xs) = f x (foldr f z xs) 
\end{minted}
\end{enumerate}
\end{Exercise}
\begin{Answer}
  Le code suivant contient les réponses des deux questions.
  
  \cinput{exercices/pstructurée_superhero.c}
\end{Answer}