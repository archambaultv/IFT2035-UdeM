\begin{Exercise}

\begin{enumerate}

\item Nommer et expliquer deux types de problèmes qui peuvent survenir
  lorsque le programmeur gère lui-même la mémoire avec
  \codeinline{malloc} et \codeinline{free}

\item Expliquer en quoi consiste la technique de comptage de références pour
gérer la mémoire. Nommer et expliquer une faiblesse importante de
cette méthode.

\item Nous avons vu en cours deux algorithmes pour la gestion automatique de
la mémoire : Mark and Sweep et Stop and Copy. Expliquer à l'aide d'un
pseudo-code ou d'un schéma le fonctionnement d'un de ces deux
algorithmes. Nommez un avantage de l'algorithme choisi par rapport à
l'autre.

\end{enumerate}

\end{Exercise}

\begin{Answer}
  \begin{enumerate}
  \item \begin{description}
    \item[Pointeur fou] L’objet pointé n’existe plus
    \item[Fuite mémoire] L’objet pointé n’est plus nécessaire, mais la
      mémoire n'est pas libérée. Par exemple, le pointeur n'est
      plus sur la pile mais l'objet est toujours dans le tas et donc
      innaccessible.
    \end{description}

  \item À chaque objet est associé un compteur qui indique combien de
    pointeurs existent. Lorsque le compteur passe à 0, on peut
    désallouer l’objet. Cette technique ne permet pas de gérer les
    références circulaires.

  \item \begin{description}
    \item[Mark \& Sweep] Ne déplace pas les objets en mémoire et ne
      nécessite donc pas deux tas comme dans le stop \& copy.  Puisque
      les objets ne sont pas déplacés, il est possible d'intégré cette
      technique dans un langage qui possèdent des pointeurs et qui
      permet d'obtenir l'adresse mémoire d'un objet.
      
      \begin{minted}{text}
marknSweep(){
  roots = getRoots();
  for root in roots
  mark(root);
  sweep();
}

mark(obj){
  if (obj->marked) return;
  obj->marked = True;
  for c in obj.children
  mark(c);
}

sweep(){
  h = heapStartPointer;
  do {
    if (h->marked)
      h->marked = False;
    else
      free(h);
  } while (ptr = next_object(h))
}
      \end{minted}

      \item[Stop \& copy] Déplace les objets en mémoire. Permet donc
        d'obtenir une mémoire compacte. L'allocation de nouveaux
        objets se fait rapidement, car nul besoin de chercher un
        espace assez grand dans le tas comme le ferait mark \& sweep.
    \end{description}
    \end{enumerate}
\end{Answer}