\begin{Exercise}
  \label{ex:prolog_regex}
En Prolog, tout comme en Haskell, les chaînes de caractères sont
représentées par des \emph{listes} de caractères.

Soit la définition suivante du prédicat de filtrage
\verb+match(RE,Str,Tail)+ qui dit que l'expression régulière \texttt{RE}
filtre la chaîne de caractères \texttt{Str} avec un résidu \texttt{Tail}.

\begin{minted}{prolog}
  match(RE, Str, Tail) :- append(RE, Tail, Str).
  match(any, [_|Tail], Tail).
  match(concat(RE1, RE2), Str, Tail) :-
     match(RE1, Str, Tail1), match(RE2, Tail1, Tail).
  match(repeat(RE), Str, Tail) :-
     match(RE, Str, Tail1), match(repeat(RE), Tail1, Tail).
  match(repeat(_), Str, Str).
\end{minted}
%%
Par exemple:
\begin{minted}{prolog}
  | ?- match("hello", "hello world", X).
  X = " world"
  | ?- match(concat("hel", "lo "), X, "").
  X = "hello "
  | ?- match(concat(repeat(any), concat("or", repeat(any))),
             "hello world", X).
  X = ""
  X = "d"
  X = "ld"
\end{minted}

\begin{enumerate}
\item Montrer le (les) arbres de recherche de la requête: \\
  \verb+match(concat(repeat("a"),"ab"),"abc",X)+
\item Que se passe-t-il avec la requête: \\
  \verb+match(concat(repeat("a"), "ab"), X, "")+
\item Changer le code de manière à éviter ce problème.
\item Ajouter du code pour l'expression régulière \texttt{or(RE1,RE2)}, de
  sorte à pouvoir faire des choses telles que:
\begin{minted}{prolog}
  | ?- match(or("hello", "world"), X, "").
  X = "hello"
  X = "world"
  | ?- match(repeat(or("a", "b")), "abac", X).
  X = "c"
  X = "ac"
  X = "bac"
  X = "abac"
\end{minted}
\item Montrer le (les) arbres de recherche de la requête: \\
  \verb+match(repeat(or("a", "b")), "abac", X)+
\item Ajouter du code pour l'expression régulière \texttt{and(RE1,RE2)}, de
  sorte à pouvoir faire des choses telles que:
\begin{minted}{prolog}
  | ?- match(concat(and(concat(repeat(any),
                               concat("ol",repeat(any))),
                        concat(repeat(any),
                               concat("ll",repeat(any)))),
                    "o"),
             "bell hollow world", X).
  X = "rld"
  X = "rld"
  X = "w world"
  X = "w world"
\end{minted}

\end{enumerate}
\end{Exercise}

\begin{Answer}[ref={ex:prolog_regex}]
  \schemeinput{exercices/prolog_regex.pl}
\end{Answer}