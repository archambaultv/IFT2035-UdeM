\begin{Exercise}[title={Gestion mémoire manuelle}]
\label{ex:gestion_memoire_manuelle}
Écrire une librairie de listes simplement chaînées en C.

\begin{minted}{C}
typedef struct list_elem list_elem;
struct list_elem {
  void *value;
  list_elem *next;
}
typedef struct list list;
struct list { list_elem *head; }

list *list_alloc  (void);
void  list_insert (list *l, void *v);
void *list_get    (list *l, int n);
\end{minted}

\begin{enumerate}
\item Écrire le code des trois fonctions proposées.  Compléter en ajoutant
  les opérations suivantes: \codeinline{list_remove}, \codeinline{list_free}.

\item Décrire précisément les conditions nécessaires (que l'utilisateur de
  la librairie doit suivre) pour qu'il n'y ait pas de déréférence de
  pointeur fou, ni de fuite.

\item Expliquer pourquoi ces conditions sont nécessaires et suffisantes pour
  garantir l'absence d'erreurs de type pointeur fou ou fuite.
\end{enumerate}
\end{Exercise}

\begin{Answer}
  \cinput{exercices/gestion_mémoire_manuelle.c}
\end{Answer}